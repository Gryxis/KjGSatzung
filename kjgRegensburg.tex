\documentclass[12pt]{report}
\usepackage[a4paper, left=2cm,top=2cm]{geometry}

%keine einzelnen Zeilen am Anfang/Ende:
\widowpenalty = 10000
\clubpenalty = 10000
%------------------------------------
%   Required Packages
%------------------------------------
\usepackage[table,xcdraw]{xcolor} %table, xcdraw are needed for table colors
\usepackage{enumitem} %needed to style tables
\usepackage{float} %needed for positioning tables and text in correct order
\usepackage{graphicx}
\usepackage[pages=some]{background}
\usepackage[utf8]{inputenc}
\usepackage[english,ngerman]{babel} % deutsche Bezeichnungen und Worttrennung… 
\usepackage[T1]{fontenc} % Wird u.a. f\"ur das Trennen von W\"ortern  mit Umlauten genutzt.
\usepackage{tgbonum} %schriftarten
\usepackage{ragged2e}
\usepackage[hidelinks]{hyperref} %Hyperref für navigation

%------------------------------------------
%    Farben
%------------------------------------------

\definecolor{weiß}{RGB}{255,255,255}
\definecolor{kjggrau}{RGB}{119,120,123}
\definecolor{kjghellgrau}{RGB}{192,192,192}
\definecolor{kjgblau}{RGB}{54,95,145}
\definecolor{kjgtuerkis}{RGB}{0, 181,190}
%------------------------------------------
% Schriftart und Indent
%------------------------------------------
\renewcommand{\familydefault}{\sfdefault}
\setlength\parindent{0pt}

%------------------------------------------
% Start-cover als Einband 
%------------------------------------------
\backgroundsetup{
 scale=1,
 angle=0,
 color=white,
 placement=top,
 opacity = 1,
 contents= {
  \includegraphics[width=\paperwidth, height=26.80cm]{satzung_einband.png}
 }
}

%--------------------------------------
% Fußnoten mehfrach referenzieren:
% Quelle:
% https://www.macuser.de/threads/latex-fussnote-mehrfach-referenzieren.519258/
\newcommand{\footnoteremember}[2]{%
  \footnote{#2}
  \newcounter{#1}
  \setcounter{#1}{\value{footnote}}
}
\newcommand{\footnoterecall}[1]{%
  \footnotemark[\value{#1}]
}
%--------------------------------------

%--------------------------------------
% Add a counter to number the tables for the changes
% Source:
% https://tex.stackexchange.com/questions/503668/increment-a-counter
\newcounter{tablecounter}
%Define a counter to automatically increase and display the counter
\newcommand\showcounter{\addtocounter{tablecounter}{1}\thetablecounter}
%--------------------------------------
% Verstecke manche Sachen vor dem Inhaltsverzeichnis
%--------------------------------------

\newcommand{\nocontentsline}[3]{}
\newcommand{\tocless}[2]{\bgroup\let\addcontentsline=\nocontentsline#1{#2}\egroup}

%--------------------------------------

%--------------------------------------
% Verstecke die Kapitelnamen
\usepackage{titlesec}
\titleformat{\chapter}{\bfseries\Large}{\thechapter}{1.5ex}{}
%--------------------------------------
\title{Satzung KjG Regensburg}
\author{}
\setcounter{chapter}{-1}

% durchegehnde fussnoten
\usepackage{chngcntr}
\counterwithout{footnote}{chapter}
\begin{document}
 
  \begin{center}
  %Einband
    \pagestyle{empty}
    \BgThispage
    \vspace*{\fill}
    \begin{Huge}
      { \color{kjgtuerkis}  Stand November 2023}
    \end{Huge}
    \vspace*{\fill}
    \newpage
  %Zitat Thoms Morus
    \pagestyle{empty}
    \vspace*{\fill}
   { \fontfamily{ptm}\selectfont
     \begin{Huge}
     "Tradition ist nicht das Halten der Asche, sondern das Weitergeben der Flamme"
     \end{Huge}
    }
    \newline
    \begin{large}
      - Thomas Morus -
    \end{large}
    \vspace*{\fill}
  \end{center}

  %Inhaltsverzeichnis:
  %\setcounter{tocdepth}{3}
  \tableofcontents
  \begin{flushleft}
   \it{ Bei dem Verdacht auf sexualisierte Gewalt in der Kinder- und Jugendarbeit wird auf den entsprechenden
        Anhang am Ende der Satzung verwiesen}
  \end{flushleft}
  \newpage  
  \setcounter{page}{1}


%alles in blocksatz und keine einrückung bei Paragraphen aber mit Platz dazwischen:
\begin{justify}
\setlength\parindent{0pt}
\setlength{\parskip}{\baselineskip}
%Jetzt geht es mit Inhalt los:

\chapter{Grundlagen und Ziele der Katholischen jungen Gemeinde}
In der Katholischen jungen Gemeinde (KjG) schließen sich junge Christ*innen zusammen
(Mitglied der KjG kann jede*r werden, die*der die Grundlagen und Ziele des Verbandes bejaht.).
Demokratisch und gleichberechtigt wählen alle Mitglieder altersunabhängig die Leitungen und
entscheiden über Inhalte und Arbeitsformen des Verbandes.
Ihre jeweiligen Bedürfnisse und Interessen bestimmen das verbandliche Leben. Die Gruppen,
Projekte und offenen Angebote der KjG bieten Raum für Begegnungen und Beziehungen,
gemeinsame Erlebnisse und gemeinsames Handeln. In ihnen erfahren Kinder, Jugendliche und junge
Erwachsene, dass sie ernst genommen werden und nicht allein stehen.

Die KjG unterstützt sie darin, ihr Leben verantwortlich zu gestalten und eigene Lebensperspektiven zu entwickeln.
Sie begleitet sie bei der Suche nach tragfähigen Lebensentwürfen und nach Orientierung.
Sie ermöglicht ihnen einen Zugang zum christlichen Glauben und ermutigt sie zu einem selbstverantworteten religiösen Leben.

Die KjG fördert auf vielfältige Weise, soziale, pädagogische und politische Verantwortung zu
übernehmen und unterstützt die Entwicklung persönlicher Interessen und Fähigkeiten.

Die KjG greift die Fragen und Anliegen von Kindern, Jugendlichen und jungen Erwachsenen auf
und befähigt sie, sich in Kirche und Gesellschaft zu vertreten. Insbesondere setzt sie sich dafür
ein, dass Kinder, Jugendliche und junge Erwachsene Pfarr- und Kommunalgemeinde gleichberechtigt mitgestalten können.
Sie engagiert sich für Strukturen, die Mitbestimmung und Mitentscheidung ermöglichen.

Der Zusammenschluss in der KjG schafft Voraussetzungen für eine wirksame Interessenvertretung in der Öffentlichkeit.
Die KjG arbeitet darüber hinaus mit den Mitgliedsverbänden im BDKJ
sowie mit anderen Verbänden und Organisationen zusammen.

Mit ihrem Engagement steht die KjG für eine demokratische, gleichberechtigte und solidarische
Gesellschaft und Kirche. Sie wendet sich gegen jede Art von Ausgrenzung und Unterdrückung von
Menschen und gegen die Zerstörung der natürlichen Lebensgrundlagen.

Die KjG setzt sich ein für eine Politik, die sich orientiert an der weltweiten Verwirklichung gleicher
und gerechter Lebensbedingungen für alle Menschen und an einer
ökologisch verantworteten Lebensweise.

In diesem Anliegen erklären sich die Mitglieder der KjG solidarisch mit anderen Kindern, Jugendlichen und jungen Erwachsenen.
Sie suchen sowohl im eigenen Land als auch über Ländergrenzen
hinweg die partnerschaftliche Zusammenarbeit und Begegnung mit ihnen.

So versteht sich die KjG als Kirche in der Lebenswelt von Kindern, Jugendlichen und jungen Erwachsenen.


\chapter{{\color{red} Allgemeine Regelungen zur Satzung}}

\section{{\color{red} Geschlechterdefinitionen innerhalb der Katholischen jungen Gemeinde } \label{geschlechterdefinition}}

{\color{red} Geschlechtergerecht im Rahmen dieser Satzung bedeutet: Gremien (und Ämter) werden mit männlichen und
weiblichen Personen paritätisch besetzt.
Bei Gremien mit einer Größe von 2 Personen werden 2 Personen unterschiedlicher Geschlechterkategorie eingerichtet.
Bei Gremien mit einer Größe von bis zu 10 Personen wird zusätzlich
eine, bei mehr als 10 Personen zwei Stellen für diverse Personen eingerichtet.

Die folgenden Geschlechterkategorien finden in der KjG Anwendung:
}

\begin{itemize}
    \item {\color{red} Weiblich im Rahmen dieser Satzung bezeichnet Personen, die sich als tendenziell weiblich
          identifizieren, z.B. cis, trans* und inter* Frauen.}
    \item {\color{red} Männlich im Rahmen dieser Satzung bezeichnet Personen, die sich als tendenziell männlich
          identifizieren, z.B. cis, trans* und inter* Männer.}
    \item {\color{red} divers im Rahmen dieser Satzung bezeichnet Personen, die sich als nicht oder nicht nur weiblich
          und nicht oder nicht nur männlich identifizieren oder genderfluid sind.}
\end{itemize}


\section{{\color{red} Befähigung zur Geistlichen Leitung} \label{geistlicheleitung}}
{\color{red}
Das Amt der Geistlichen Leitung kann von Personen wahrgenommen werden,
die eine theologische oder religionspädagogische Ausbildung abgeschlossen haben.
}

\section{\color{red} Beschränkt Geschäftsfähig \label{geschaeftsfaehigkeit}}
§106 BGB: Ein Minderjähriger, der das siebente Lebensjahr vollendet hat, ist nach Maße der §107 bis §113 in
der Geschäftsfähigkeit beschränkt.



\chapter{KjG in der Pfarrgemeinde/Ortsgruppe}
Eine Ortsgruppe ist wie eine Pfarrgemeinschaft zu behandeln, jedoch ist sie nicht an eine Pfarrei
gebunden. Im Folgenden wurde auf die explizite Nennung der Ortsgruppe verzichtet.
\section{Mitglieder}
Mitglied in der KjG kann jede*r werden, die*der die Grundlagen und Ziele des Verbandes bejaht.

Die*der Einzelne wird Mitglied der KjG Pfarrgemeinschaft, indem sie*er das erklärt und die Pfarrleitung diese Erklärung annimmt.
Das Mitglied ist grundsätzlich verpflichtet, den Mitgliedsbeitrag fristgerecht zu bezahlen.
Die Höhe des jeweiligen Mitgliedsbeitrags wird von der jeweiligen
Ebene festgelegt und erhoben.

Besteht keine Anbindung an eine Pfarrgemeinschaft oder Ortsgruppe, kann der*die Einzelne
die Mitgliedschaft gegenüber dem Diözesanverband erklären. Diese Erklärung wird wirksam,
wenn sie von der Diözesanleitung angenommen wird.

Eine Mitgliedschaft in der KjG kann als aktive oder passive Mitgliedschaft erworben werden.

Die Mitgliedschaft erlischt durch Austritt, Ausschluss oder Tod. Der Austritt ist für das folgende Jahr
schriftlich gegenüber der Pfarrleitung bzw. Diözesanleitung bis zum 31.12. des laufenden Jahres zu erklären.

Über den Ausschluss eines Mitglieds entscheidet die Pfarrleitung bzw. Diözesanleitung nach Anhörung der*des Betroffenen.
Das betroffene Mitglied kann gegen diesen Beschluss bei der
Mitgliederversammlung bzw. Diözesankonferenz Berufung einlegen. Diese entscheidet verbindlich.
\subsection{Aktive Mitgliedschaft}
Als aktives Mitglied nimmt sie*er an einer oder mehreren der angebotenen Gesellungs- und Arbeitsformen teil.

Durch die aktive Mitgliedschaft in der KjG haben Mitglieder ein Recht auf Mitbestimmung
sowie die Chance auf Aus­ und Weiterbildung. Sie können Verantwortung übernehmen und selbst
Angebote schaffen.

Jedes aktive Mitglied ist stimmberechtigt und wählbar.
\subsection{Passive Mitgliedschaft}
Passive Mitgliedschaften in der Katholischen jungen Gemeinde dienen der ideellen und/oder finanziellen Unterstützung der Arbeit des Verbandes.
Der Mitgliedsbeitrag verbleibt bei der jeweiligen Ebene.

Die passive Mitgliedschaft schließt eine Stimmberechtigung in der Katholischen jungen
Gemeinde aus. Mitglieder einer passiven Mitgliedschaft dürfen nicht gewählt werden.

Passive Mitglieder zählen nicht in die Stimmschlüsselberechnung hinein.
\section{Die Pfarrgemeinschaft}
Die Mitglieder der Katholischen jungen Gemeinde in der Pfarrei bilden die KjG Pfarrgemeinschaft.

Sie ist Mitglied im Diözesanverband der KjG. Sie arbeitet mit anderen
BDKJ-Jugendverbänden zusammen und kann mit diesen den BDKJ bilden.

Sie führt den Namen „Katholische junge Gemeinde (KjG) Pfarrgemeinschaft/Ortsgruppe N.N.“.
Das Verbandszeichen ist der Seelenbohrer.

Die KjG Pfarrgemeinschaft bestimmt nach demokratischen Regeln im Rahmen der Grundlagen
und Ziele sowie der Satzung, Leitung, Aufgaben, Gesellungs- und Arbeitsformen entsprechend
der örtlichen Situation.

Die Leiter*innen der Teams, Gruppen, Clubs oder Arbeitskreise werden entweder von den Mitgliedern
der jeweiligen Gesellungs- bzw. Arbeitsform gewählt oder durch die Pfarrleitung nach
Anhörung der Pädagogischen Leitungsrunde berufen.

Die KjG Pfarrgemeinschaft führt an den Diözesanverband einen Beitrag ab, dessen Höhe von der
Diözesankonferenz beschlossen wird. Die KjG Pfarrgemeinschaft kann einen Pfarrbeitrag erheben,
dessen Höhe in der Mitgliederversammlung beschlossen wird.

Die Vertretung im Diözesanverband erfolgt durch die Pfarrleitung oder deren Delegierte.
\subsection{Satzung der Pfarrgemeinschaft}
Die KjG Pfarrgemeinschaft kann sich im Rahmen der Grundlagen und Ziele, sowie der Satzung
des Diözesanverbandes eine eigene Pfarrsatzung geben. Existiert keine eigene Satzung, gilt die
der nächsthöheren Ebene. Diese Satzung muss mindestens enthalten:
\begin{itemize}
  \item Anerkennung und Verpflichtung auf die Grundlagen und Ziele der KjG
  \item Die Mitgliedschaft im Diözesanverband
  \item Die Zugehörigkeit zum BDKJ
\end{itemize}

Und gemäß der nachfolgenden Paragraphen:
\begin{itemize}
  \item Die Mitgliederversammlung
  \item Die Pfarrleitung
  \item Den Kindersenat
\end{itemize}

Diese Satzung kann gemäß der nachfolgenden Paragraphen enthalten:
\begin{itemize}
  \item Das Orga-Team
  \item Die Pädagogische Leitungsrunde
\end{itemize}

Die Satzung bedarf der Zustimmung durch die Diözesanleitung.\\

Gegen die Entscheidung der Diözesanleitung kann beim Diözesanausschuss Einspruch erhoben
werden. Der Diözesanausschuss entscheidet abschließend.
\subsection{Ausschluss der Pfarrgemeinschaft}
Über den Ausschluss einer KjG Pfarrgemeinschaft entscheidet die Diözesanleitung nach Anhörung
der Betroffenen und der zuständigen Arbeitsgemeinschaftsleitung.
Diese Anhörung geschieht in einer außerordentlichen Mitgliederversammlung. Die betroffene KjG Pfarrgemeinschaft
kann gegen diesen Beschluss beim Diözesanausschuss Berufung einlegen. 
Der Diözesanausschuss entscheidet abschließend.

\subsection{Auflösung der Pfarrgemeinschaft}
Zu einer Auflösungsversammlung der KjG Pfarrgemeinschaft muss mindestens 14 Tage vorher
schriftlich eingeladen werden. Der Einladung ist eine ausführliche Begründung beizufügen. Der
Auflösung der KjG Pfarrgemeinschaft müssen drei Viertel der anwesenden stimmberechtigten
Mitglieder zustimmen.

Das Vermögen der KjG Pfarrgemeinschaft fällt bei Auflösung an den Diözesanverband.
Dieser ist verpflichtet, das Vermögen der KjG Pfarrgemeinschaft zweckgebunden zu verwalten.
Dies gilt im Falle eines Ausschlusses sinngemäß für Vermögen aus öffentlichen Bezuschussungen.
Sollte sich die KjG Pfarrgemeinschaft innerhalb von drei Jahren neu konstituieren, ist ihr das
Vermögen auszuhändigen. Ist dies nicht der Fall, fällt das verwaltete Vermögen an den Diözesanverband.

\section{Die Organe der KjG Pfarrgemeinschaft}
Die Organe der KjG Pfarrgemeinschaft sind die Mitgliederversammlung, die Pfarrleitung,
die Pädagogische Leitungsrunde, das Orga-Team und der Kindersenat.

\subsection{Die Mitgliederversammlung}
Die Mitgliederversammlung ist das oberste beschlussfassende Organ der KjG Pfarrgemeinschaft.
Sie trifft im Rahmen der Grundlagen und Ziele, sowie der Satzung des Diözesanverbandes und
der Beschlüsse der Diözesankonferenz die grundlegenden Entscheidungen über die Arbeit der
KjG Pfarrgemeinschaft.

\subsubsection{Aufgaben der Mitgliederversammlung}
Der Mitgliederversammlung sind insbesondere folgende Aufgaben vorbehalten:
\begin{itemize}
  \item Beratung und Beschlussfassung über
    \begin{itemize}  
      \item die an die Mitgliederversammlung gerichteten Anträge
      \item die Finanzen der KjG Pfarrgemeinschaft
      \item die Pfarrsatzung
      \item die Jahresplanung
      \item den Pfarrbeitrag
    \end{itemize}
  \item Entgegennahme des Jahresberichtes der Pfarrleitung
  \item Entgegennahme des Kassenberichtes
  \item Entlastung der Pfarrleitung
  \item Wahl der Pfarrleitung
  \item Wahl der Kassenprüfenden
  \item Wahl des Kindersenates
  \item Abwahl einzelner Mitglieder der Pfarrleitung
\end{itemize}

\subsubsection{Zusammensetzung der Mitgliederversammlung}
Stimmberechtigte Mitglieder der Mitgliederversammlung sind:
\begin{itemize}
  \item Die Mitglieder der KjG Pfarrgemeinschaft, sofern sie den Mitgliedsbeitrag für das laufende Jahr bezahlt haben
\end{itemize}
Beratende Mitglieder der Mitgliederversammlung sind:
\begin{itemize}
  \item Ein*e Hauptamtliche*r der Pfarrei
  \item Ein Mitglied des Sachausschuss Jugend der Pfarrei
  \item Ein Mitglied des Kreisvorstandes des BDKJ
  \item Ein Mitglied der Leitung der zuständigen Arbeitsgemeinschaft der KjG
  \item Ein*e Vertreter*in des Diözesanverbandes der KjG
  \item Die nicht stimmberechtigten Mitglieder der KjG Pfarrgemeinschaft
\end{itemize}
\subsubsection{Einberufung, Ablauf und Beschlussfähigkeit der Mitgliederversammlung}
Die Mitgliederversammlung findet wenigstens einmal jährlich statt. Sie wird von der Pfarrleitung
drei Wochen vorher unter Bekanntgabe der Tagesordnung und der Frist für die Einreichung
der Wahlvorschläge einberufen. Jedes Mitglied wird auf geeignete Weise eingeladen.
Eine Mitgliederversammlung muss einberufen werden, wenn die Pädagogische Leitungsrunde,
der Kindersenat oder 1/5 der Mitglieder dies beantragen. Anträge können vor und während der 
Mitgliederversammlung eingebracht werden.

Anträge auf Abwahl der Pfarrleitung und Anträge auf Satzungsänderung sind den Mitgliedern
der Mitgliederversammlung mindestens 14 Tage vor dem Termin der Mitgliederversammlung mit
Begründung zuzuleiten.

Die Mitgliederversammlung beschließt und wählt mit einfacher Mehrheit der anwesenden, stimmberechtigten Mitglieder.
Stimmenthaltungen bleiben unberücksichtigt. Abstimmungen über
Abwahl der Pfarrleitung und Änderung der Satzung bedürfen der 2/3-Mehrheit der anwesenden,
stimmberechtigten Mitglieder. Für den Ablauf der Mitgliederversammlung gilt im
Übrigen die Geschäftsordnung der Mitgliederversammlung, wie sie im Anhang dieser Satzung zu finden ist, sinngemäß. Über die Mitgliederversammlung
wird Protokoll geführt und den Mitgliedern zugänglich gemacht.

Die Mitgliederversammlung ist beschlussfähig, wenn frist- und formgerecht eingeladen wurde.
\subsection{Die Pfarrleitung}
\subsubsection{Aufgaben der Pfarrleitung}
Die Pfarrleitung ist verantwortlich für die Leitung und Vertretung \footnote{ vgl. §26 BGB} der KjG Pfarrgemeinschaft im Rahmen der Grundlagen und Ziele sowie der Satzung und der Beschlüsse der Organe der Pfarrgemeinde und der nächsthöheren Ebene. Dabei ist jedes Mitglied der Pfarrleitung alleine vertretungsberechtigt.

Ihre Aufgaben sind insbesondere:
\begin{itemize}
  \item Einberufung, Vorbereitung und Leitung der Mitgliederversammlung
  \item Einberufung, Vorbereitung und Leitung der Pädagogischen Leitungsrunde, des Orga-Teams und des Kindersenates
  \item Sorge für die Durchführung der Beschlüsse der Mitgliederversammlung
  \item Vertretung und Mitarbeit auf der Diözesanebene der KjG
  \item Vertretung der KjG Pfarrgemeinschaft in Kirche und Öffentlichkeit
  \item Zusammenarbeit mit anderen KjG Pfarrgemeinschaften
  \item Zusammenarbeit mit den anderen BDKJ Jugendverbände
  \item Zusammenarbeit mit den in den Pfarreien tätigen Gemeinschaften und Gremien
  \item Verantwortung für die Finanzen
  \item Sorge um die Aus- und Weiterbildung der Mitarbeitenden durch den Verband (insbesondere der Gruppenleiter*innen)
  \item Einbringen von Spiritualität
\end{itemize}

\subsubsection{Zusammensetzung der Pfarrleitung}

Die Pfarrleitung ist {\color{red} geschlechtergerecht (\ref{geschlechterdefinition}) } zu
besetzen, ihr gehören mindestens an:

Stimmberechtigt:
\begin{itemize}
  \item 5 Pfarrleiter*innen (2 männlich, 2 weiblich, 1 divers)
  \item 2 Geistliche Leiter*innen {\color{red} (\ref{geistlicheleitung}) unterschiedlicher Geschlechterkategorie}
\end{itemize}
Die Aufgaben der Pfarrleitung können auch wahrgenommen werden, wenn nicht alle Stellen besetzt sind.
Von der Verpflichtung zur {\color{red}geschlechtergerechten (\ref{geschlechterdefinition}) }
Besetzung sind die KjG Pfarrgemeinschaften ausgenommen, in
denen nur Personen {\color{red}einer Geschlechterkategorie} vertreten sind.

Mindestens ein Mitglied der Pfarrleitung muss voll geschäftsfähig sein.\\
Für mindestens die Hälfte der Stellen müssen {\color{red}  beschränkt geschäftsfähige Personen (\ref{geschaeftsfaehigkeit})}
zur Wahl zugelassen werden.

Die Pfarrleitung kann für die Kassenführung eine*n oder mehrere Kassierer*innen berufen.
Mit dem Amt des*der Kassiers*in sind keine zusätzlichen Stimmrechte verbunden.\\

Die Mitglieder der Pfarrleitung werden von der Mitgliederversammlung für
zwei Jahre gewählt.
Die Mitglieder der Pfarrleitung können ihren Rücktritt nur
gegenüber der Mitgliederversammlung erklären.

Sind alle Stellen der Pfarrleitung vakant, so dürfen deren Aufgaben von der
Diözesanleitung übernommen werden. In diesem Fall hat die Diözesanleitung die Möglichkeit eine
Stimme bei der Mitgliederversammlung wahrzunehmen.

\subsection{Das Orga-Team}

\subsubsection{Aufgaben des Orga-Teams}

Das Orga-Team berät im Rahmen der Beschlüsse der Mitgliederversammlung die Arbeit der KjG
Pfarrgemeinschaft und stimmt die Interessen der einzelnen Gesellungs- und Arbeitsformen ab,
unbenommen der Letztverantwortung der Pfarrleitung.

Dem Orga-Team sind insbesondere folgende Aufgaben vorbehalten:
\begin{itemize}
  \item Planung und Sorge für die Durchführung der Veranstaltungen und Aktionen der KjG Pfarrgemeinschaft
  \item Gewinnung von Leiter*innen und freien Mitarbeitenden
\end{itemize}

\subsubsection{Zusammensetzung und Einberufung des Orga-Teams}
Mitglied des Orga-Teams kann jedes Mitglied der Pfarrgemeinschaft werden. Das Orga-Team
trifft sich nach Bedarf und wird von der Pfarrleitung einberufen und geleitet.

\subsection{Die Pädagogische Leitungsrunde}

\subsubsection{Aufgaben der Pädagogischen Leitungsrunde}
Die Pädagogische Leitungsrunde dient den Leiter*innen der einzelnen Gesellungs- und Arbeitsformen als Ort für:
\begin{itemize}
  \item Erfahrungsaustausch
  \item Weiterbildung
  \item Informationen über die Situation der Kinder und Jugendlichen in der Pfarrgemeinde
  \item Reflexion der Gruppenarbeit und des eigenen Leitungsverhaltens
\end{itemize}

\subsubsection{Zusammensetzung und Einberufung der Pädagogischen Leitungsrunde}
Zur Pädagogischen Leitungsrunde gehören:
\begin{itemize}
  \item Die Pfarrleitung
  \item Die Leiter*innen der einzelnen Gesellungs- und Arbeitsformen
\end{itemize}

Gäste können von der Pädagogischen Leitungsrunde eingeladen werden.

Die Pädagogische Leitungsrunde wird regelmäßig, mindestens viermal im Jahr,
von der Pfarrleitung einberufen und geleitet.

\subsection{Der Kindersenat}
Der Kindersenat dient der Kindermitbestimmung in der Zeit zwischen den Mitgliederversammlungen.
In den Kindersenat können Dauermitglieder der KjG Pfarrgemeinschaft gewählt werden,
die zum Zeitpunkt der Wahl unter 13 Jahre alt sind.

Die Mitglieder des Kindersenats werden auf der Mitgliederversammlung von den unter 13 Jahre alten
Dauermitgliedern für die Dauer von einem Jahr gewählt.

\subsubsection{Aufgaben des Kindersenates}
Zu den Aufgaben des Kindersenats gehören:
\begin{itemize}
  \item Anliegen von Kindern in Pfarrleitungs- und Pädagogischer Leitungsrunde einbringen
  \item Beratende Funktion bei Aktionen und Veranstaltungen für Kinder in der KjG Pfarrgemeinschaft
\end{itemize}

\subsubsection{Zusammensetzung und Einberufung des Kindersenats}

Der Kindersenat ist {\color{red} geschlechtergerecht (\ref{geschlechterdefinition}) } zu besetzen,
ihm gehören mindestens an:

\begin{itemize}
  \item 2 männliche Kinder
  \item 2 weibliche Kinder
  \item 1 diverses Kind
\end{itemize}
Die Aufgaben des Kindersenates können auch dann wahrgenommen werden, wenn nicht alle Ämter besetzt sind.
Von der Verpflichtung zur {\color{red}geschlechtergerechten (\ref{geschlechterdefinition}) } Besetzung sind die KjG Pfarrgemeinschaften ausgenommen,
in denen nur Personen {\color{red} einer Geschlechterkategorie} vertreten sind.

Der Kindersenat wird regelmäßig, mindestens zweimal im Jahr, von der Pfarrleitung einberufen
und von einem Mitglied der Pfarrleitung geleitet.

\chapter{KjG auf mittlerer Ebene}

\section{KjG Arbeitsgemeinschaften}

Die KjG Pfarrgemeinschaften des Diözesanverbandes können zur besseren Wahrnehmung ihrer
Aufgaben auf der mittleren Ebene Arbeitsgemeinschaften bilden.

Sie führt den Namen „Katholische junge Gemeinde (KjG) Arbeitsgemeinschaft N.N.“.
Das Verbandszeichen ist der Seelenbohrer.

Vordringliche Aufgabe der Arbeitsgemeinschaft ist die Unterstützung, Förderung und Koordinierung
der Arbeit der KjG Pfarrgemeinschaften.

Die Arbeitsgemeinschaft hat keine Beitragshoheit.

Alle beteiligten KjG Pfarrgemeinschaften müssen der Arbeitsgemeinschaft im Rahmen der Grundlagen
und Ziele der KjG, sowie der Satzung der KjG Diözesanverband Regensburg eine eigene
Satzung geben. Die Satzungsgebung muss einstimmig auf einer Konferenz der beteiligten Pfarreien
beschlossen werden. Satzungsänderungen sind dann mit einer 2/3 Mehrheit möglich.

Die Satzung muss enthalten:
\begin{itemize}
  \item Anerkennung und Verpflichtung auf die Grundlagen und Ziele der KjG
  \item Die Mitgliedschaft im KjG Diözesanverband Regensburg
  \item Die Zugehörigkeit zum BDKJ
  \item Eine mindestens jährlich stattfindende Konferenz der beteiligten Pfarrgemeinschaften, bei der
        die Geschäftsordnung der Diözesankonferenz der KjG Diözesanverband Regensburg gilt
  \item Die Wahl einer {\color{red} geschlechtergerecht (\ref{geschlechterdefinition}) } zu besetzenden Leitung
\end{itemize}

Die Satzung bedarf der Zustimmung der Diözesanleitung. Gegen die Entscheidung der Diözesanleitung
kann beim Diözesanausschuss Einspruch erhoben werden. Der Diözesanausschuss entscheidet
abschließend. Zur Satzungsgebung ist die Diözesanleitung
anzuhören.

Der Arbeitsgemeinschaftskonferenz sind insbesondere folgende Aufgaben vorbehalten:
\begin{itemize}
  \item Erfahrungsaustausch und Koordinierung der Arbeit der beteiligten KjG Pfarrgemeinschaften
  \item Beratung der Arbeit des Diözesanverbandes
  \item Beratung und Beschlussfassung über Veranstaltungen und Aktionen der Arbeitsgemeinschaft
  \item Planung von Schulungen für die Verantwortlichen der KjG Pfarrgemeinschaften
  \item Beratung und Beschlussfassung über die Finanzen der Arbeitsgemeinschaft
  \item Entgegennahme des Berichtes der Arbeitsgemeinschaftsleitung
\end{itemize}
\section{KjG Bezirksverbände}
Der Diözesanverband kann sich in Bezirksverbände gliedern. Dafür gelten die entsprechenden
Bestimmungen der Satzung des Bundesverbandes.
\chapter{KjG in der Diözese}
Der Diözesanverband der Katholischen jungen Gemeinde ist der Zusammenschluss der
KjG Pfarrgemeinschaften in der Diözese.

Der Diözesanverband ist Mitglied im Bundesverband der Katholischen jungen Gemeinde und im
Diözesanverband des BDKJ.

Er führt den Namen „Katholische junge Gemeinde (KjG) Diözesanverband Regensburg“,
auch kurz KjG Diözesanverband Regensburg, mit Sitz in Regensburg.
Das Verbandszeichen ist der Seelenbohrer.

Der Diözesanverband ist ein nicht rechtsfähiger Verein.

Aufgabe des Diözesanverbandes ist die Unterstützung, Förderung und Koordinierung der Arbeit
der KjG Pfarrgemeinschaften und der Arbeitsgemeinschaften der KjG Pfarrgemeinschaften und
deren Vertretung in Kirche und Gesellschaft.

\section{Gemeinnützigkeit}

\tocless\subsection{}
Der KjG Diözesanverband Regensburg verfolgt ausschließlich und unmittelbar gemeinnützige
und kirchliche Zwecke im Sinne des Abschnitts ,,Steuerbegünstigte Zwecke`` der Abgabenordnung.

\tocless\subsection{}
\label{subsec:Zweck}
Zweck des KjG Diözesanverbandes Regensburg ist die
Förderung der Religion (§ 52 Abs. 2 S. 1 Nr. 2 AO),
der Jugendhilfe (§ 52 Abs. 2 S. 1 Nr. 4 AO),
der Erziehung, Volks- und Berufsbildung einschließlich
der Studentenhilfe (§ 52 Abs. 2 S. 1 Nr.7 AO), der internationalen Gesinnung
und des Völkerverständigungsgedankens (§ 52 Abs. 2 S. 1 Nr. 13 AO), des
bürgerlichen Engagements zugunsten gemeinnütziger und kirchlicher Zwecke
(§ 52 Abs. 2 S. 1 Nr. 25 AO) sowie die Verfolgung kirchlicher Zwecke (§ 54 AO).

\tocless\subsection{}
Der Satzungszweck wird verwirklicht insbesondere durch:
\begin{itemize}
  \item die Wahrnehmung kirchlicher Kinder- und Jugendarbeit insbesondere in der Diözese Regensburg
        in Einheit mit der Gesamtkirche und in Übereinstimmung mit den Grundrechten selbst,
  \item die Schaffung von Begegnungsmöglichkeiten im Rahmen der Organisation oder Durchführung von
        Begegnungs- und Bildungsmaßnahmen sowie Aktionen,
  \item die Förderung demokratischen, gleichberechtigten und solidarischen Engagements, das sich
        gegen jede Art von Ausgrenzung oder Unterdrückung von Menschen wendet,
  \item die Förderung einer ökologisch verantworteten Lebensweise um die Zerstörung der
          natürlichen Lebensgrundlage einzudämmen,
  \item die nationale und internationale Zusammenarbeit um partnerschaftlich und solidarisch für
        eine weltweite Etablierung von gleichen und gerechten Lebensbedingungen für alle Menschen einzustehen,
  \item die Schaffung von Raum für Kinder und Jugendliche sowie junge Erwachsene und deren
        Gruppierungen:
    \begin{itemize}
      \item um Begegnungen und Beziehungen zu fördern und durch gemeinsame Erlebnisse und gemeinsames Handeln
            das Zugehörigkeitsgefühl und die Glaubensgemeinschaft zu stärken
      \item zur ständigen Wertorientierung und Wertschätzung innerhalb der Gruppierung und der Kirche
      \item zur Standortüberprüfung und Entwicklung von Lebensperspektiven in Einheit mit einem 
            selbstverantworteten religiösen Lebens
      \item zur Ermutigung soziale, politische und pädagogische Verantwortung zu übernehmen und  
            persönliche Interessen und Fähigkeiten zu entwickeln
      \item zur Schaffung von Impulsen und Möglichkeiten zur Entwicklung eines demokratischen Zusammenwirkens
            und Handelns in Einheit mit der Gesamtkirche und in Übereinstimmung 
            mit den Grundrechten in einer globalisierten Welt.
    \end{itemize}
\end{itemize}
\tocless\subsection{}
Der KjG Diözesanverband Regensburg darf seinen Satzungszweck auch durch Hilfspersonen (§ 57 Abs. 1 S. 2 AO) verwirklichen.
\tocless\subsection{}
Der KjG Diözesanverband Regensburg ist selbstlos tätig; er verfolgt nicht in erster Linie eigenwirtschaftliche Zwecke.
\tocless\subsection{}
Mittel des KjG Diözesanverbandes Regensburg dürfen nur für die satzungsmäßigen Zwecke
verwendet werden. Die Mitglieder erhalten keine Zuwendungen aus Mitteln der Körperschaft.
\tocless\subsection{}
Es darf keine Person durch Ausgaben, die dem Zweck des KjG Diözesanverbandes Regensburg
fremd sind, oder durch unverhältnismäßig hohe Vergütungen begünstigt werden.
\section{Die Organe des Diözesanverbandes}
Die Organe des KjG Diözesanverbandes Regensburg sind die Diözesankonferenz, der Diözesanausschuss und die
Diözesanleitung.

\subsection{Die Diözesankonferenz}
Die Diözesankonferenz ist das oberste beschlussfassende Organ des KjG Diözesanverbandes Regensburg. Sie
trifft im Rahmen der Grundlagen und Ziele, sowie der Satzung des Bundesverbandes und der
Beschlüsse der Bundeskonferenz die grundlegenden Entscheidungen über die Arbeit des Diözesanverbandes.
\subsubsection{Aufgaben der Diözesankonferenz}
Der Diözesankonferenz sind insbesondere folgende Aufgaben vorbehalten:
\begin{itemize} 
  \item Beschlussfassung über:
    \begin{itemize} 
      \item die Diözesansatzung
      \item den Diözesanbeitrag
      \item die Jahresplanung
      \item das Schulungsprogramm
      \item gemeinsame Aktionen
      \item die Einrichtung und Auflösung von diözesanen Teams und Arbeitsgruppen
    \end{itemize}
  \item Entgegennahme der Tätigkeitsberichte der Diözesanleitung und des Diözesanausschusses
  \item Entgegennahme des Finanzberichtes
  \item Entlastung der Diözesanleitung
  \item Wahl:
    \begin{itemize}
      \item der Diözesanleitung
      \item des Diözesanausschusses
      \item des Wahlausschusses
      \item der Kassenprüfung
      \item der Delegierten für
            \begin{itemize}
              \item die Bundeskonferenz
              \item die Bundesräte
              \item die Mitgliederversammlungen der Bundesstelle der
                    Katholischen jungen Gemeinde e.V., sofern die
                    Diözesanleitung nicht ausreichend besetzt ist
              \item die Diözesanversammlungen des BDKJ
              \item die Mitgliederversammlung der KjG Landesstelle e.V., sofern
                    die Diözesanleitung unbesetzt ist
              \item die Landesversammlung und Landesausschüsse der KjG
                    Landesarbeitsgemeinschaft Bayern
              \item das Diözesankomitee Regensburg
            \end{itemize}
            \item ggf. der Mitglieder von Sachausschüssen
    \end{itemize}
  \item Abwahl einzelner Mitglieder der Diözesanleitung beziehungsweise des Diözesanausschusses
\end{itemize}
\subsubsection{Ausschüsse}
Die Diözesankonferenz kann für bestimmte Aufgaben
{\color{red} geschlechtergerecht (\ref{geschlechterdefinition}) } besetzte Sachausschüsse einrichten.
Sachausschüsse zu {\color{red}geschlechterkategoriespezifischen} Belangen sind hiervon ausgenommen.

Den Vorsitz der Sachausschüsse hat ein Mitglied der Diözesanleitung inne, dieser kann delegiert
werden.

Der Wahlausschuss leitet die Wahlen. Der Wahlausschuss ist {\color{red}geschlechtergerecht (\ref{geschlechterdefinition}) } zu besetzen. Den Vorsitz
des Wahlausschusses hat ein Mitglied der Diözesanleitung inne, dieser kann delegiert werden.

\subsubsection{Zusammensetzung der Diözesankonferenz}
Stimmberechtigte Mitglieder der Diözesankonferenz sind:
\begin{itemize}
  \item 2 Delegierte pro KjG Pfarrgemeinschaft
  \item Die Mitglieder der Diözesanleitung
\end{itemize}

Die Delegation ist folgendermaßen zu besetzen:
\begin{itemize}
  \item{ 2 Mitglieder der Pfarrleitung bzw. von Pfarrleitung oder Mitgliederversammlung
        Delegierte {\color{red}unterschiedlicher Geschlechterkategorien}}
\end{itemize}

Von der Verpflichtung zur  {\color{red} geschlechtergerechten (\ref{geschlechterdefinition}) } Besetzung
sind die KjG Pfarrgemeinschaften ausgenommen,
in denen nur Personen {\color{red}einer Geschlechterkategorie} vertreten sind.

Hat eine KjG Pfarrgemeinschaft bis drei Wochen vor der Diözesankonferenz nicht
die Mitgliedsbeiträge des Vorjahres an den Diözesanverband bezahlt, so ruht ihr
Stimmrecht.\\

Sollte die Diözesankonferenz in der zweiten Hälfte des Jahres stattfinden, so muss eine KjG Pfarrgemeinschaft zusätzlich zum selben Zeitpunkt mindestens 35 Prozent der Mitgliedsbeiträge des aktuellen Jahres an den Diözesanverband gezahlt haben, sonst ruht ihr Stimmrecht ebenso.
Wenn das Stimmrecht einer KjG Pfarrgemeinschaft ruht, so bedeutet das, dass die von ihr Delegierten nicht stimmberechtigt sind. Diese gelten im Sinne der Satzung als beratende Mitglieder.

Beratende Mitglieder sind:
\begin{itemize}
  \item Die Diözesanreferent*innen
  \item Die Mitglieder des Diözesanausschusses
  \item Ein Mitglied von Sachausschüssen und diözesanen Projektgruppen
  \item Ein Mitglied der Bundesleitung der Katholischen jungen Gemeinde
  \item Ein*e Vertreter*in des Landesvorstandes der KjG-Landesarbeitsgemeinschaft Bayern
  \item Ein Mitglied des BDKJ Diözesanvorstandes
  \item Der*die Vorsitzende des Vereins zur Förderung der Katholischen jungen Gemeinde in der
        Diözese Regensburg e.V.
  \item Je ein Mitglied der diözesanen Teams und
        Arbeitsgruppen\footnoteremember{Dauermitglied}{Das jeweilige Mitglied muss
        Dauermitglied im KjG Diözesanverband Regensburg sein}
  \item Je ein Mitglied der Leitung der Arbeitsgemeinschaften der
        Pfarreien\footnoterecall{Dauermitglied}
\end{itemize}

Gäste können von der Diözesanleitung eingeladen werden.
\subsubsection{Einberufung und Ablauf der Diözesankonferenz}
Die Diözesankonferenz tritt mindestens einmal jährlich zusammen und wird von der Diözesanleitung
einberufen und geleitet. Sie ist in der Regel öffentlich.
Eine außerordentliche Diözesankonferenz muss einberufen werden, wenn der Diözesanausschuss oder ein
Drittel der Pfarrgemeinschaften dies beantragen.

Der Ablauf der Diözesankonferenz regelt sich nach der Geschäftsordnung.

\subsubsection{Änderung der Satzung des Diözesanverbandes}
Änderungen der Diözesansatzung können nur beschlossen werden, wenn zwei Drittel der anwesenden
stimmberechtigten Mitglieder zustimmen und der Änderungsantrag den Mitgliedern der
Diözesankonferenz mindestens drei Wochen vorher schriftlich mitgeteilt worden ist.

\subsection{Der Diözesanausschuss}
Der Diözesanausschuss berät im Rahmen der Grundlagen und Ziele und der Beschlüsse der Diözesankonferenz
über die Arbeit und beschließt über laufende wichtige Angelegenheiten des KjG Diözesanverbandes Regensburg.

\subsubsection{Aufgaben des Diözesanausschusses}
Dem Diözesanausschuss sind insbesondere folgende Aufgaben vorbehalten:
\begin{itemize}
  \item Planung und Vorbereitung der Diözesankonferenz
  \item Sorge für die Durchführung der Beschlüsse der Diözesankonferenz
  \item Beschlussfassung über den Etat des Diözesanverbandes
  \item Schlichtung und Entscheidung bei Konfliktfällen
        \footnote{Betroffene Mitglieder haben bei der Entscheidung kein Stimmrecht}
  \item Pflegen der Kontakte zu den KjG Pfarrgemeinschaften
\end{itemize}

\subsubsection{Zusammensetzung des Diözesanausschusses}
Stimmberechtigte Mitglieder des Diözesanausschusses sind:
\begin{itemize}
  \item 4 weibliche Mitglieder der Pfarrleitungen bzw. Mitglieder einer Pfarrgemeinschaft, die von der
        Mitgliederversammlung ein Mandat erhalten haben.
  \item 4 männliche Mitglieder der Pfarrleitungen bzw. Mitglieder einer Pfarrgemeinschaft, die von
        der Mitgliederversammlung ein Mandat erhalten haben.
  \item 1 diverses Mitglied der Pfarrleitungen bzw. Mitglied einer
  Pfarrgemeinschaft, das von der Mitgliederversammlung ein Mandat erhalten
  hat.
  \item Die Mitglieder der Diözesanleitung
\end{itemize}

Von diesen Personen, die nicht zur Diözesanleitung gehören, sind bis zu zwei
Personen {\color{red} Geistliche Leitung (\ref{geistlicheleitung})
unterschiedlicher Geschlechterkategorie}.

Beratende Mitglieder sind:
\begin{itemize}
  \item Die Diözesanreferent*innen
\end{itemize}

Die Aufgaben des Diözesanausschusses können auch dann wahrgenommen werden, wenn nicht
alle Stellen besetzt sind.

Mitglied im Diözesanausschuss können Personen werden, die mindestens {\color{red} beschränkt geschäftsfähig (\ref{geschaeftsfaehigkeit})} sind.
Von den stimmberechtigten Mitgliedern des Diözesanausschusses, die nicht Teil der Diözesanleitung sind,
muss aber mindestens eine Person, unabhängig {\color{red}der Geschlechterkategorie}, voll geschäftsfähig sein.

Gäste können von der Diözesanleitung oder dem Diözesanausschuss eingeladen werden.

Die Vertretungen der Pfarrgemeinschaften werden von der Diözesankonferenz für zwei Jahre
gewählt. Die Wahl ist persönlich; eine Vertretung im Diözesanausschuss ist nicht möglich. Mit
dem Wegfall der Voraussetzung für den Diözesanausschuss erlischt die Mitgliedschaft im Diözesanausschuss.

\subsubsection{Einberufung und Ablauf des Diözesanausschusses}
Der Diözesanausschuss tritt nach Bedarf, mindestens jedoch zweimal jährlich zusammen. Er wird
von der Diözesanleitung mindestens 14 Tage vorher einberufen. Den Vorsitz hat die Diözesanleitung.

\subsection{Die Diözesanleitung}

\subsubsection{Aufgaben der Diözesanleitung}
Die Diözesanleitung führt die Geschäfte und vertritt den KjG Diözesanverband Regensburg in sämtlichen
Angelegenheiten gerichtlich und außergerichtlich.

Zu den Aufgaben der Diözesanleitung gehören insbesondere:
\begin{itemize}
  \item Leitung und Geschäftsführung des Diözesanverbandes im Rahmen der Grundlagen und Ziele
        des Verbandes, sowie der Satzung und der Beschlüsse der Organe des Bundes- und Diözesanverbandes
  \item Vertretung des Diözesanverbandes im Bundesverband
  \item Vertretung des Diözesanverbandes im BDKJ auf Diözesanebene
  \item Vertretung des Diözesanverbandes in der Landesarbeitsgemeinschaft der KjG
  \item Vertretung des Diözesanverbandes in Kirche und Gesellschaft
\end{itemize}

Zur Erfüllung ihrer Aufgaben kann die Diözesanleitung mit Zustimmung des Diözesanausschusses
Referent*innen, Sachbearbeitende sowie Mitarbeitende berufen.

\subsubsection{Zusammensetzung der Diözesanleitung}
Zur Diözesanleitung gehören:
\begin{itemize}
	\item 3 weibliche Mitglieder,
	\item 3 männliche Mitglieder,
	\item 1 diverses Mitglied.
\end{itemize}

Von diesen Personen sind bis zu zwei Personen {\color{red} Geistliche Leitung (\ref{geistlicheleitung})
unterschiedlicher Geschlechterkategorie.}

Die Aufgaben der Diözesanleitung können auch dann wahrgenommen werden, wenn nicht alle
Ämter besetzt sind. Mindestens ein Mitglied der Diözesanleitung muss voll geschäftsfähig sein.

Kann eine Stelle der Geistlichen Leitung nicht besetzt werden, kann eine weitere Diözesanleitung
gewählt werden. Kann keine der beiden Geistlichen Leitungsstellen besetzt werden, entscheidet
die Diözesankonferenz, welche Position bis zur nächsten Wahl unbesetzt bleibt.

Die Diözesanleitung wird von der Diözesankonferenz für zwei Jahre gewählt. Die Mitglieder der
Diözesanleitung können ihren Rücktritt nur vor der Diözesankonferenz erklären.

Den Mitgliedern der Diözesanleitung werden die bei der Verbandsarbeit entstandenen, angemessenen Auslagen
ersetzt. Mitglieder der Diözesanleitung können darüber hinaus eine angemessene Vergütung erhalten. Nach Prüfung der Gründe beschließt der Diözesanausschuss die Höhe der Vergütung. Prüfung und Beschluss müssen bis zum Ende der Amtszeit der Diözesanleitung erfolgen.

\subsubsection{Kontakt zu KjG Pfarrgemeinschaften}
Die Wahrnehmung der Kontakte zu den KjG Pfarrgemeinschaften ist Aufgabe von Diözesanleitung
und gewähltem Diözesanausschuss. Bei Bedarf können weitere interessierte KjG Mitglieder,
vorzugsweise mit Erfahrung in der KjG Pfarreiarbeit, mit dieser Aufgabe betraut werden.

\section{Auflösung des Diözesanverbandes}
Zu einer Auflösungsversammlung des KjG Diözesanverbandes Regensburg muss mindestens 28 Tage vorher
schriftlich eingeladen werden. Der Einladung ist eine Begründung hinzuzufügen. Drei Viertel der
anwesenden stimmberechtigten Mitglieder müssen der Auflösung zustimmen. Das weitere Vorgehen
im Falle der Auflösung regelt die Satzung des Bundesverbandes.

Bei Auflösung oder Aufhebung des KjG Diözesanverbandes Regensburg oder bei Wegfall steuerbegünstigter Zwecke fällt das Vermögen des Diözesanverbandes an den Bundesstelle der KjG e.V. (Carl-Mosters-Platz 1 - 40477 Düsseldorf), der es unmittelbar und ausschließlich für gemeinnützige oder kirchliche Zwecke im Sinne der Nr. 3.1.2 zu verwenden hat.

\chapter{Schlussbestimmungen}
Die Neufassung der Satzung tritt nach ihrer Beschlussfassung durch die große
Diözesankonferenz der Katholischen jungen Gemeinde Diözesanverband Regensburg
2023 und nach Zustimmung durch die Bundesleitung der KjG in Kraft.




\part*{Anhänge}
\addcontentsline{toc}{part}{Anhänge}

\chapter*{Geschäftsordnung der Diözesankonferenz}
\addcontentsline{toc}{chapter}{Geschäftsordnung der Diözesankonferenz}

\subsection*{§1 Termin}
Der Termin der jährlichen Diözesankonferenz wird von der Diözesankonferenz beschlossen.
\subsection*{§2 Vorbereitung}
Die Vorbereitung der Diözesankonferenz erfolgt durch den Diözesanausschuss.
\subsection*{§3 Vorläufige Tagesordnung}
Die Vorläufige Tagesordnung der Diözesankonferenz wird im Diözesanausschuss beraten und beschlossen.
\subsection*{§4 Tagungsform}
Die Diözesankonferenz kann auf einzelfallbezogenen Beschluss auch über Wege der elektronischen Kommunikation (z.B. per Telefon- oder Videokonferenz) tagen. Mischformen sind zulässig. Der entsprechende Beschluss wir durch die
Diözesankonferenz selbst oder den Diözesanausschuss getroffen.
\subsection*{§5 Einberufung}
Die Diözesankonferenz wird von der Diözesanleitung acht Wochen vor dem festgelegten Termin einberufen.
\subsection*{§6 Öffentlichkeit}
Die Diözesankonferenz ist öffentlich. Die Öffentlichkeit kann durch Beschluss aufgehoben werden.
Personaldebatten sind nicht öffentlich. Bei Personaldebatten sind nur die stimmberechtigten und beratenden
Mitglieder der Diözesankonferenz anwesend.
\subsection*{§7 Stellvertretung}
Die stimmberechtigten Mitglieder können sich bei der Diözesankonferenz vertreten lassen. Die Vertretung
der Delegierten bedarf der Zustimmung der Pfarrleitung. Frauen können nur durch Frauen, Männer nur durch
Männer und diverse Delegierte nur durch diverse Personen vertreten werden. Die Vereinigung mehrerer Stimmen auf eine Person ist unzulässig.
\subsection*{§8 Leitung}
Die Leitung der Diözesankonferenz obliegt der Diözesanleitung. Sie bestimmt, welches Mitglied den Vorsitz
führt. Sie kann den Vorsitz delegieren. Die*der jeweilige Vorsitzende kann sich an den Beratungen nicht beteiligen.
Wenn sie*er das Wort ergreifen will, muss der Vorsitz an andere Personen abgegeben werden.
Die*der Vorsitzende kann jederzeit das Wort zu einer Feststellung ergreifen.
\subsection*{§9 Anträge}
Anträge an die Diözesankonferenz können von Mitgliedern, diözesanen Teams und Ausschüssen gestellt werden.

Anträge sind mit Begründung bis spätestens sechs Wochen vor Beginn der Diözesankonferenz bei der Diözesanleitung
schriftlich einzureichen und den Mitgliedern der Diözesankonferenz drei Wochen vorher zuzuleiten.

Es gibt folgende Sonderformen mit diesen Regularien:
\begin{itemize}
  \item Satzungsänderungsantrag: vgl. regulärer Antrag
  \item Antrag auf Abwahl von einzelnen Diözesanleitungs- bzw. Diözesanausschussmitgliedern:
        vgl. regulärer Antrag
  \item Initiativantrag: kann jederzeit gestellt werden, bedarf zur Aufnahme in die Tagesordnung der Zustimmung
        eines Drittels der anwesenden stimmberechtigten Mitglieder der Diözesankonferenz
  \item Änderungsantrag zu bestehendem Antrag: kann jederzeit gestellt werden
\end{itemize}

\subsection*{§10 Unterlagen}
Drei Wochen vor Beginn erhalten die Mitglieder der Diözesankonferenz durch die Diözesanleitung die notwendigen
Unterlagen und zwar:
\begin{itemize}
  \item die vorläufige Tagesordnung
  \item die Anträge mit Begründung
  \item die Berichte der Diözesanleitung 
  \item die Berichte des Diözesanausschusses 
  \item die Berichte der diözesanen Teams
\end{itemize}

\subsection*{§11 Beschlussfähigkeit}
Die Diözesankonferenz ist beschlussfähig, wenn ordnungsgemäß eingeladen wurde und wenigstens 50 Prozent
der stimmberechtigten Mitglieder anwesend sind sowie {\color{red} keine Geschlechterkategorie} mehr als 75 Prozent der anwesenden stimmberechtigten Mitglieder ausmacht.

Die Diözesankonferenz gilt als beschlussfähig, solange die Beschlussunfähigkeit nicht ausdrücklich
festgestellt wird. Ist die Beschlussunfähigkeit festgestellt kann außer
der Schließung der Konferenz kein Beschluss gefasst werden. Die Beratungen können aber gemäß der Tagesordnung und den durch die Geschäftsordnung festgelegten Bestimmungen fortgesetzt werden. Solange die Diözesankonferenz nicht geschlossen wurde, kann zu einem späteren Zeitpunkt erneut die Beschlussfähigkeit festgestellt werden.

\subsection*{§12 Beginn der Beratungen}
Die Beratungen beginnen mit der Feststellung der Beschlussfähigkeit und der Feststellung der endgültigen
Tagesordnung sowie des Zeitplans.

Auf Antrag können Tagesordnungspunkte aufgenommen, umgestellt oder abgesetzt werden.
\subsection*{§13 Schluss der Beratungen}
Die Diözesankonferenz kann die Beratungen vertagen oder schließen. Beschlüsse zum Vertagen oder Schließen
der Diözesankonferenz bedürfen der Zwei-Drittel-Mehrheit. Die Abstimmung über den Schlussantrag ist
nur zulässig, wenn wenigstens ein Mitglied die Gelegenheit erhält, dagegenzusprechen. Der Schlussantrag
geht dem Vertagungsantrag und dieser allen übrigen Anträgen vor.
\subsection*{§14 Beratungen}
Das Wort wird durch die*den Vorsitzende*n in der Reihenfolge des Eingangs der Wortmeldungen erteilt.
Es werden {\color{red}nach Geschlechterkategorie getrennte} Redner*innenlisten geführt. Diese Listen werden im Wechsel aufgerufen.
Berichte werden abschnittsweise beraten.

Antragsteller*innen und Berichterstatter*innen können außerhalb der Reihenfolge das Wort verlangen. Die Redezeit
kann von der*dem Vorsitzenden begrenzt werden. Dies kann von der Diözesankonferenz durch Mehrheitsbeschluss
aufgehoben werden. Die*der Vorsitzende kann Redner*innen, die nicht zur Sache sprechen, das
Wort entziehen. Gegen Maßnahmen der*des Vorsitzenden ist Widerspruch möglich. Über den Widerspruch
entscheidet die Diözesankonferenz.

\subsection*{§15 Wortmeldungen zur Geschäftsordnung}
Zu Anträgen oder Äußerungen zur Geschäftsordnung kann jederzeit das Wort verlangt werden.
Durch Anträge zur Geschäftsordnung wird die Redner*innenliste unterbrochen. Die Anträge sind sofort zu
behandeln. Anträge und Äußerungen zur Geschäftsordnung dürfen sich nur mit dem Gang der Verhandlungen
befassen; das sind:

\begin{itemize}
  \item Antrag auf Schluss der Debatte und sofortige Abstimmung
  \item Antrag auf Schluss der Redner*innenliste
  \item Antrag auf Beschränkung der Redezeit
  \item Antrag auf Vertagung eines Antrages oder eines Tagesordnungspunktes
  \item Antrag auf Unterbrechung der Sitzung
  \item Antrag auf Nichtbefassung
  \item Hinweis zur Geschäftsordnung
  \item Antrag auf Überweisung an einen Ausschuss
\end{itemize}

Erhebt sich bei einem Antrag zur Geschäftsordnung kein Widerspruch, ist der
Antrag angenommen; andernfalls ist nach Anhörung eines*r Gegenredners*in sofort
abzustimmen.

Über die Auslegung der Wortmeldungen zur Geschäftsordnung entscheidet die*der Vorsitzende verbindlich.


\subsection*{§16 Persönliche Erklärung}
Nach Schluss der Beratung eines Tagesordnungspunktes oder nach Beendigung der Abstimmung kann die*der
Vorsitzende das Wort zu einer persönlichen Bemerkung oder Erklärung erteilen. Diese muss schriftlich bei
der*dem Protokollführenden abgegeben werden. Eine Debatte hierüber findet nicht statt.
\subsection*{§17 Abstimmungen}
Die Abstimmung erfolgt mit einfacher Mehrheit der anwesenden stimmberechtigten Mitglieder.
Stimmengleichheit gilt als Ablehnungen. Enthaltungen werden nicht gezählt. Überwiegen die Enthaltungen die 
Ja-Stimmen, so muss die Diskussion über den Beratungsgegenstand auf Antrag neu eröffnet und erneut
abgestimmt werden.

Abstimmungen über Änderungen der Satzung und der Geschäftsordnung bedürfen der
Zwei-Drittel-Mehrheit der anwesenden stimmberechtigten Mitglieder. Abgestimmt wird mit Stimmkarten.
Auf Antrag muss geheim abgestimmt werden.
Liegen zu einem Beratungsgegenstand mehrere Anträge vor, so ist über den weitestgehenden zuerst abzustimmen.

Unmittelbar nach einer Abstimmung kann bei begründeten Zweifeln an der Richtigkeit der Abstimmung 
Wiederholung verlangt werden.

Auf Antrag kann im Verlauf der Beratungen über Beschlüsse noch einmal abgestimmt werden.
Die*der Vorsitzende stellt das Ergebnis der Abstimmung fest und verkündet es.

\subsection*{§18 Wahlen}
Wahlen werden grundsätzlich in geheimer Abstimmung durchgeführt. Auf Antrag kann Abstimmung mit
Stimmkarten und/oder en bloc erfolgen, wenn sich kein Widerspruch ergibt.

Gewählt wird mit einfacher Mehrheit. Bei Stimmengleichheit erfolgt Stichwahl.

Der Wahl voraus geht eine Personalbefragung und auf Antrag eine Personaldebatte. Bei Wahlen für den 
Diözesanausschuss und für Sachausschüsse der Diözesankonferenz gilt: Die jeweils kandidierenden Personen
sind gewählt, wenn sie die meistgenannten Kandidat*innen sind und wenn diese Nennungen mindestens ein
Drittel der Stimmen ausmachen.

\subsection*{§19 Wahl der Mitglieder der Diözesanleitung}
Zur Vorbereitung der Wahl bildet die Diözesankonferenz einen Wahlausschuss. Aufgabe des Wahlausschusses
ist es, der Diözesankonferenz geeignete Kandidat*innen für die Wahl vorzuschlagen und die Wahl zu leiten.
Vorschlagsrecht haben alle stimmberechtigten Mitglieder der Diözesankonferenz.

Die dem Wahlausschuss bekannten Kandidat*innen sind den Mitgliedern der Diözesankonferenz drei Wochen
vorher zu benennen. Der Wahl geht eine Personalbefragung und eine Personaldebatte voraus.

Gewählt ist, wer im ersten Durchgang mehr als 50 Prozent der abgegebenen gültigen Stimmen auf sich 
vereinigen kann. Wer mehr als zwei Drittel Neinstimmen erhält, ist von den folgenden Wahlgängen ausgeschlossen. 
Im zweiten Wahlgang genügt die einfache Stimmenmehrheit. Sind mehr als 50 Prozent der abgegebenen
gültigen Stimmen Enthaltungen, so ist die*der Kandidat*in nicht gewählt.

Über jede*n Kandidat*in wird mit Ja, Nein oder Enthaltung abgestimmt. Es dürfen nur so viele Ja-Stimmen
abgegeben werden, wie Ämter zu besetzen sind. Steht für ein Amt nur ein*e Kandidat*in zur Verfügung, so ist für
die Wahl die absolute Mehrheit der abgegebenen gültigen Stimmen der Anwesenden erforderlich.

\subsection*{§20 Abwahl von einzelnen Mitgliedern der Diözesanleitung bzw. des Diözesanausschusses}
Anträge auf Abwahl von einzelnen Mitgliedern der Diözesankonferenz bzw. des Diözesanausschusses sind
bis spätestens sechs Wochen vor Beginn der Diözesankonferenz der Diözesanleitung schriftlich einzureichen
und vier Wochen vorher von der Diözesanleitung den Mitgliedern der Diözesankonferenz zuzuleiten.

Zur Abwahl von Diözesanleitungsmitgliedern bzw. Diözesanausschussmitgliedern ist eine Mehrheit von zwei
Drittel der abgegebenen gültigen Stimmen notwendig.

\subsection*{§21 Protokoll}
Über jede Diözesankonferenz wird ein Ergebnisprotokoll angefertigt, das von der Diözesanleitung 
unterschrieben wird. Dieses Protokoll enthält die Namen der anwesenden Mitglieder, die Tagesordnung,
die gefassten Beschlüsse im Wortlaut mit Abstimmungsergebnis und alle ausdrücklich zum Zwecke der 
Niederschrift abgegebenen Erklärungen.
\subsection*{§22 Genehmigung des Protokolls}
Das Protokoll wird allen Mitgliedern der Diözesankonferenz innerhalb von acht Wochen zugeschickt. Es gilt
als genehmigt, wenn innerhalb von sechs Wochen nach Zustellung bei der Diözesanleitung gegen die Fassung
des Protokolls schriftlich kein Einspruch erhoben wird.

Die Diözesanleitung benachrichtigt die Mitglieder der Diözesankonferenz über Einsprüche gegen das Protokoll.
Über Annahme oder Ablehnung entscheidet der Diözesanausschuss.
\subsection*{§23 Außerordentliche Diözesankonferenz}
Eine außerordentliche Diözesankonferenz muss einberufen werden, wenn der
Diözesanausschuss oder ein Drittel der Pfarrleitungen dies beantragen. Eine explizite Tagungsform kann gewünscht werden, der Diözesanausschuss prüft, ob diese in vertretbarer Zeit realisierbar ist und entscheidet abschließend darüber. Er ist angehalten dem Wunsch nachzugehen. Die Einberufung zu einer außerordentlichen Diözesankonferenz muss mindestens sechs Wochen vor dem Termin mit Bekanntgabe der Tagesordnung erfolgen. Die Diözesanleitung muss eine beantragte außerordentliche Diözesankonferenz spätestens vier Wochen nach der Beantragung einberufen.
\subsection*{§24 Abweichung von der Geschäftsordnung}
Von der Geschäftsordnung kann im Ausnahmefall an einzelnen Punkten mit einer Zwei-Drittel-Mehrheit der
anwesenden stimmberechtigten Mitglieder abgewichen werden.
\subsection*{§25 Schlussbestimmungen}
Die Neufassung der Geschäftsordnung tritt nach ihrer Beschlussfassung durch die Diözesankonferenz der
Katholischen Jungen Gemeinde Diözesanverband Regensburg 2001 und nach Zustimmung durch die Bundesleitung
der KJG in Kraft.

\chapter*{Geschäftsordnung der Mitgliederversammlung}
\addcontentsline{toc}{chapter}{Geschäftsordnung der Mitgliederversammlung}

\subsection*{§1 Termin}
Der Termin der jährlichen Mitgliederversammlung wird von der Mitgliederversammlung beschlossen.
\subsection*{§2 Vorbereitung}
Die Vorbereitung der Mitgliederversammlung erfolgt durch die Pfarrleitung.
\subsection*{§3 Vorläufige Tagesordnung}
Die vorläufige Tagesordnung der Mitgliederversammlung wird von der Pfarrleitung beraten und beschlossen.
\subsection*{§4 Tagungsform}
Die Mitgliederversammlung kann auf einzelfallbezogenen Beschluss auch über Wege der elektronischen Kommunikation (z.B. per Telefon- oder Videokonferenz) tagen. Mischformen sind zulässig. Der entsprechende Beschluss wird durch die Mitgliederversammlung selbst oder die Pfarrleitung, nach Möglichkeit in Absprache mit der pädagogischen Leitungsrunde, getroffen.
\subsection*{§5 Einberufung}
Die Mitgliederversammlung wird von der Pfarrleitung mindestens drei Wochen vor dem festgelegten Termin
unter Bekanntgabe der Tagesordnung und der Frist für die Einreichung der Wahlvorschläge einberufen.
\subsection*{§6 Öffentlichkeit}
Personaldebatten sind nicht öffentlich. In Personaldebatten sind nur die stimmberechtigten Mitglieder der
Mitgliederversammlung anwesend. Alle, die im jeweiligen Wahlgang kandidieren, müssen die Personaldebatte
verlassen.
\subsection*{§7 Leitung}
Die Leitung der Mitgliederversammlung obliegt der Pfarrleitung. Sie bestimmt, welches Mitglied den Vorsitz
führt. Sie kann den Vorsitz delegieren. Der*die jeweilige Vorsitzende kann sich an den Beratungen nicht
beteiligen. Wenn er*sie das Wort ergreifen will, muss der Vorsitz an andere Personen abgegeben werden.
Der*die Vorsitzende kann jederzeit das Wort zu einer Feststellung ergreifen.
\subsection*{§8 Anträge}
Anträge an die Mitgliederversammlung können von einzelnen Mitgliedern, der Pfarrleitung, dem Orga-Team,
der Pädagogischen Leitungsrunde und dem Kindersenat gestellt werden. Die Anträge mit Begründungen
können vor und während der Mitgliederversammlung gestellt werden. Anträge, die während der
Mitgliederversammlung gestellt werden, bedürfen zur Aufnahme in die Tagesordnung der Zustimmung eines Drittels
der anwesenden stimmberechtigten Mitglieder der Diözesankonferenz.

Anträge auf Abwahl der Pfarrleitung und Anträge auf Satzungsänderung sind den Mitgliedern der
Mitgliederversammlung mindestens 14 Tage vor dem Termin der Mitgliederversammlung mit Begründung schriftlich
zuzuleiten.
\subsection*{§9 Beschlussfähigkeit}
Die Mitgliederversammlung ist beschlussfähig, wenn ordnungsgemäß und fristgerecht eingeladen wurde.
\subsection*{§10 Beginn der Beratungen}
Die Beratungen beginnen mit der Feststellung der Beschlussfähigkeit und der Feststellung der endgültigen
Tagesordnung sowie des Zeitplans. Auf Antrag können Tagesordnungspunkte aufgenommen, umgestellt
oder gestrichen werden.
\subsection*{§11 Schluss der Beratungen}
Die Mitgliederversammlung kann die Beratungen vertagen oder schließen. Beschlüsse zum Vertagen oder
Schließen der Mitgliederversammlung bedürfen der Zwei-Drittel-Mehrheit. Die Abstimmung über den
Schlussantrag ist nur zulässig, wenn wenigstens ein Mitglied die Gelegenheit erhält, dagegenzusprechen.
Der Schlussantrag geht dem Vertagungsantrag vor und dieser allen übrigen Anträgen.
\subsection*{§12 Beratungen}
Das Wort wird durch die*den Vorsitzende*n in der Reihenfolge des Eingangs der Wortmeldungen erteilt.
Es werden {\color{red}nach Geschlechterkategorie getrennte} Redner*innenlisten geführt. Diese Listen werden im Wechsel aufgerufen. Berichte werden
abschnittsweise beraten. Antragstellende und Berichterstattende können außerhalb der Reihenfolge das
Wort verlangen. Die Redezeit kann von der*dem Vorsitzenden begrenzt werden. Dies kann von der 
Mitgliederversammlung durch Mehrheitsbeschluss aufgehoben werden. Der*die Vorsitzende kann Redenden, die
nicht zur Sache sprechen, das Wort entziehen. Gegen Maßnahmen des*der Vorsitzenden ist Widerspruch
möglich. Über den Widerspruch entscheidet die Mitgliederversammlung.
\subsection*{§13 Wortmeldungen zur Geschäftsordnung}
Zu Anträgen oder Hinweise zur Geschäftsordnung kann jederzeit das Wort verlangt werden. Durch Anträge
zur Geschäftsordnung wird die Redner*innenliste unterbrochen. Die Anträge sind sofort zu behandeln. Anträge und
Hinweise zur Geschäftsordnung dürfen sich nur mit dem Verlauf der Beratungen befassen; das sind:

\begin{itemize}
  \item Antrag auf Schluss der Debatte und sofortige Abstimmung
  \item Antrag auf Schluss der Redner*innenliste
  \item Antrag auf Beschränkung der Redezeit
  \item Antrag auf Vertagung eines Antrages oder eines Tagesordnungspunktes
  \item Antrag auf Unterbrechung der Sitzung
  \item Antrag auf Nichtbefassung
  \item Hinweis zur Geschäftsordnung
  \item Antrag auf Überweisung an einen Ausschuss
\end{itemize}

Erhebt sich bei einem Antrag zur Geschäftsordnung kein Widerspruch, ist der Antrag angenommen,
andernfalls ist nach Anhörung einer*s Gegenredenden sofort abzustimmen. Über die Auslegung der Wortmeldungen
zur Geschäftsordnung entscheidet der*die Vorsitzende verbindlich.

\subsection*{§14 Persönliche Erklärung}
Nach Schluss der Beratung eines Tagesordnungspunktes oder nach Beendigung der Abstimmung kann
die*der Vorsitzende das Wort zu einer persönlichen Bemerkung oder Erklärung erteilen. Diese muss
schriftlich bei der*dem Protokollführenden abgegeben werden. Eine Debatte hierüber findet nicht statt.
\subsection*{§15 Abstimmungen}
Die Abstimmung erfolgt mit einfacher Mehrheit der anwesenden, stimmberechtigten Mitglieder.
Stimmengleichheit gilt als Ablehnung. Enthaltungen werden nicht gezählt. Überwiegen die Enthaltungen
die Ja-Stimmen, so muss die Diskussion über den Beratungsgegenstand auf Antrag neu eröffnet und erneut
abgestimmt werden. Abstimmungen über Änderungen der Satzung und der Geschäftsordnung bedürfen der 
Zwei-Drittel-Mehrheit der anwesenden, stimmberechtigten Mitglieder. Abgestimmt wird mit Stimmkarten.
Auf Antrag muss geheim abgestimmt werden. Liegen zu einem Beratungsgegenstand mehrere Anträge vor, so ist
über den Weitestgehenden zuerst abzustimmen. Unmittelbar nach einer Abstimmung kann bei begründeten
Zweifeln an der Richtigkeit der Abstimmung Wiederholung verlangt werden. Auf Antrag kann im Verlauf der
Beratungen über Beschlüsse noch einmal abgestimmt werden. Die*der Vorsitzende stellt das Ergebnis der
Abstimmung fest und verkündet es.
\subsection*{§16 Wahlen}
Wahlen werden grundsätzlich in geheimer Abstimmung durchgeführt. Auf Antrag kann Abstimmung mit
Stimmkarten und/oder en bloc erfolgen, wenn sich kein Widerspruch ergibt. Gewählt wird mit einfacher Mehrheit. Bei
Stimmengleichheit erfolgt Stichwahl. Der Wahl voraus geht eine Personalbefragung und auf Antrag eine
Personaldebatte. Gewählt sind die meistgenannten Kandidierenden, jedoch müssen diese Nennungen mindestens
ein Drittel der abgegebenen Stimmen ausmachen.

Die Mitglieder des Kindersenats werden von den bis einschließlich 12 Jahre alten aktiven Mitgliedern gewählt.
\subsection*{§17 Abwahl von einzelnen Mitgliedern der Pfarrleitung}
Anträge auf Abwahl von einzelnen Mitgliedern der Pfarrleitung sind bis spätestens zwei Wochen vor Beginn
der Mitgliederversammlung den Mitgliedern schriftlich zuzuleiten. Zur Abwahl von Pfarrleitungsmitgliedern
ist eine Mehrheit von zwei Drittel der abgegebenen gültigen Stimmen notwendig.
\subsection*{§18 Protokoll}
Über jede Mitgliederversammlung wird ein Ergebnisprotokoll angefertigt, das von der Pfarrleitung
unterschrieben wird. Dieses Protokoll enthält die Namen der anwesenden Mitglieder, die Tagesordnung, die
gefassten Beschlüsse im Wortlaut mit Abstimmungsergebnis und alle ausdrücklich zum Zwecke der Niederschrift
abgegebenen Erklärungen.
\subsection*{§19 Genehmigung des Protokolls}
Das Protokoll wird allen Mitgliedern der Mitgliederversammlung innerhalb von acht Wochen zugeschickt. Es
gilt als genehmigt, wenn innerhalb von sechs Wochen nach Zustellung bei der Pfarrleitung gegen die Fassung
des Protokolls schriftlich kein Einspruch erhoben wird. Die Pfarrleitung benachrichtigt die Mitglieder der
Mitgliederversammlung über Einsprüche gegen das Protokoll. Über Annahme oder Ablehnung entscheidet
die Pfarrleitung.
\subsection*{§20 Außerordentliche Mitgliederversammlung}
Eine außerordentliche Mitgliederversammlung muss einberufen werden, wenn die Pädagogische Leitungsrunde, der Kindersenat oder ein Drittel der Mitglieder dies beantragen. Eine explizite Tagungsform kann gewünscht werden, die Pfarrleitung prüft, ob diese in vertretbarer Zeit realisierbar ist und entscheidet abschließend darüber. Sie ist angehalten dem Wunsch nachzugehen. Die Pfarrleitung muss eine beantragte außerordentliche Mitgliederversammlung spätestens vier Wochen nach der Beantragung einberufen.
\subsection*{§21 Abweichung von der Geschäftsordnung}
Von der Geschäftsordnung kann im Ausnahmefall an einzelnen Punkten mit einer Zwei-Drittel-Mehrheit der
anwesenden, stimmberechtigten Mitglieder abgewichen werden.
\subsection*{§22 Schlussbestimmungen}
Die Neufassung der Geschäftsordnung tritt nach ihrer Beschlussfassung durch die Diözesankonferenz der
Katholischen jungen Gemeinde Diözesanverband Regensburg 2016 und nach Zustimmung durch
die Bundesleitung der KjG in Kraft.


\chapter*{Vorgehensweise bei Verdacht auf sexualisierte Gewalt}
\addcontentsline{toc}{chapter}{Vorgehensweise bei Verdacht auf sexualisierte Gewalt}
\subsection*{ 0.}
Grundsätzlich wird der Kreis der mit dem Verdachtsfall betrauten Personen so klein wie möglich gehalten.
Aus Gründen des Opfer- und Täterschutzes werden Informationen und insbesondere Namen  
streng vertraulich behandelt.
\subsection*{1.}
Besteht der Verdacht, dass ein Mitglied sexualisierter Gewalt ausgesetzt ist, holen sich ehrenamtliche
Mitarbeiter*innen Unterstützung bei der Diözesanleitung, den Bildungsreferent*innen oder bei  
ausgewiesenen Fachberatungsstellen.
\subsection*{2.}
Besteht eine begründete Vermutung, dass ein Mitglied sexualisierte Gewalt auf andere ausübt, müssen
umgehend die Diözesanleitung oder die Bildungsreferent*innen informiert werden. Die Diözesanleitung
und die Bildungsreferent*innen klären das weitere Vorgehen mit professioneller Unterstützung.
\subsection*{3.}
„Im begründeten Verdachtsfall sind Hauptberufliche und Ehrenamtliche Mitarbeitende sofort von  
deren Tätigkeit zu entbinden.“
(Prävention sexueller Gewalt in der KjG, Grundsatzinformationen für Pfarreien der LAG Bayern)
\subsection*{4.}
Der Leitfaden der KjG LAG Bayern zur Prävention sexueller Gewalt dient als Orientierung zur Vorgehensweise.
\subsection*{5.}
Alle Schritte müssen schriftlich in einem Handlungsprotokoll festgehalten werden und von der gesamten
Diözesanleitung unterschrieben werden.
\subsection*{6.}
Der Verbandsausschluss ist als letzte Maßnahme anzusehen, die ergriffen werden kann.

\newpage
\pagestyle{empty}
\newgeometry{top=0cm,left=0cm}
\includegraphics[width=\paperwidth, height=\paperheight]{kjgsatzungende.png}
\restoregeometry
% Ende des Dokumentes
\end{justify}
\end{document}
