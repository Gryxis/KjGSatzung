\documentclass[12pt]{report}


%------------------------------------
%   Required Packages
%------------------------------------
\usepackage{xcolor}
\usepackage{graphicx}
\usepackage[pages=some]{background}
\usepackage[utf8]{inputenc}
\usepackage[english,ngerman]{babel} % deutsche Bezeichnungen und Worttrennung… 
\usepackage[T1]{fontenc} % Wird u.a. f\"ur das Trennen von W\"ortern  mit Umlauten genutzt.
\usepackage[a4paper, left=2cm,top=2cm]{geometry}
%------------------------------------
%    Farben
%------------------------------------

\definecolor{weiß}{RGB}{255,255,255}
\definecolor{kjggrau}{RGB}{119,120,123}
\definecolor{kjghellgrau}{RGB}{192,192,192}
\definecolor{kjgblau}{RGB}{54,95,145}
\definecolor{kjgtuerkis}{RGB}{0, 181,190}

\backgroundsetup{
 scale=1,
 angle=0,
 color=white,
 placement=top,
 opacity = 1,
 contents= {
  \includegraphics[width=\paperwidth, height=26.80cm]{satzung_einband.png}
 }
}

%--------------------------------------
% Fußnoten mehfrach referenzieren:
% Quelle:
% https://www.macuser.de/threads/latex-fussnote-mehrfach-referenzieren.519258/
\newcommand{\footnoteremember}[2]{
  \footnote{#2}
  \newcounter{#1}
  \setcounter{#1}{\value{footnote}}
}
\newcommand{\footnoterecall}[1]{%
  \footnotemark[\value{#1}]
}
%--------------------------------------



\title{Satzung KjG Regensburg}
\author{}

\begin{document}
  \maketitle
  \begin{center}
  %Einband
    \pagestyle{empty}
    \BgThispage
    \vspace*{\fill}
    \begin{Huge}
      { \color{kjgtuerkis}  Stand April 2020}
    \end{Huge}
    \vspace*{\fill}
    \newpage
  %Zitat Thoms Morus
    \pagestyle{empty}
    \vspace*{\fill}
    \begin{Huge}
     "Tradition ist nicht das Halten der Asche, sondern das Weitergeben der Flamme"
    \end{Huge}
    \newline
    \begin{large}
      - Thomas Morus -
    \end{large}
    \vspace*{\fill}
  \end{center}

  %Inhaltsverzeichnis:
  %\setcounter{tocdepth}{3}
  \tableofcontents
   \it{ Bei dem Verdacht auf sexualisierte Gewalt in der Kinder- und Jugendarbeit wird auf den entsprechenden Anhang am Ende der Satzung verwiesen}
  \newpage  
  \setcounter{page}{1}


%alles links zentrieren und keine einrückung bei Paragraphen:
\begin{flushleft}
%Jetzt geht es mit Inhalt los:

\chapter{Grundlagen und Ziele der Katholischen jungen Gemeinde}
In der Katholischen jungen Gemeinde (KjG) schließen sich junge Christ*innen zusammen. 
Mitglied der KjG kann jede*r werden, der*die die Grundlagen und Ziele des Verbandes bejaht.
Demokratisch und gleichberechtigt wählen alle Mitglieder altersunabhängig die Leitungen und
entscheiden über die Inhalte und Arbeitsformen des Verbandes.
Ihre jeweiligen Bedürfnisse und Interessen bestimmen das verbandliche Leben. Die Gruppen,
Projekte und offenen Angebote der KjG bieten Raum für Begegnungen und Beziehungen,
gemeinsame Erlebnisse und gemeinsames Handeln. In ihnen erfahren Kinder, Jugendliche und junge
Erwachsene, dass sie ernst genommen werden und nicht allein stehen.

Die KjG unterstützt sie darin, ihr Leben verantwortlich zu gestalten und eigene Lebensperspektiven zu entwickeln.
Sie begleitet sie bei der Suche nach tragfähigen Lebensentwürfen und nach Orientierung.
Sie ermöglicht ihnen einen Zugang zum christlichen Glauben und ermutigt sie zu einem selbstverantworteten religiösen Leben.

Die KjG fördert auf vielfältige Weise, soziale, pädagogische und politische Verantwortung zu
übernehmen und unterstützt die Entwicklung persönlicher Interessen und Fähigkeiten.

Die KjG greift die Fragen und Anliegen von Kindern, Jugendlichen und jungen Erwachsenen auf
und befähigt sie, sich in Kirche und Gesellschaft zu vertreten. Insbesondere setzt sie sich dafür
ein, dass Kinder, Jugendliche und junge Erwachsene Pfarr- und Kommunalgemeinde gleichberechtigt mitgestalten können.
Sie engagiert sich für Strukturen, die Mitbestimmung und Mitentscheidung ermöglichen.

Der Zusammenschluss in der KjG schafft Voraussetzungen für eine wirksame Interessenvertretung in der Öffentlichkeit.
Die KjG arbeitet darüber hinaus mit den Mitgliedsverbänden im BDKJ
sowie mit anderen Verbänden und Organisationen zusammen.

Mit ihrem Engagement steht die KjG für eine demokratische, gleichberechtigte und solidarische
Gesellschaft und Kirche. Sie wendet sich gegen jede Art von Ausgrenzung und Unterdrückung von
Menschen und gegen die Zerstörung der natürlichen Lebensgrundlagen.

Die KjG setzt sich ein für eine Politik, die sich orientiert an der weltweiten Verwirklichung gleicher
und gerechter Lebensbedingungen für alle Mädchen und Jungen, Frauen und Männer und an einer
ökologisch verantworteten Lebensweise.

In diesem Anliegen erklären sich die Mitglieder der KjG solidarisch mit anderen Kindern, Jugendlichen und jungen Erwachsenen.
Sie suchen sowohl im eigenen Land als auch über Ländergrenzen
hinweg die partnerschaftliche Zusammenarbeit und Begegnung mit ihnen.

So versteht sich die KjG als Kirche in der Lebenswelt von Kindern, Jugendlichen und jungen Erwachsenen.

\chapter{KjG in der Pfarrgemeinde/Ortsgruppe}
Eine Ortsgruppe ist wie eine Pfarrgemeinschaft zu behandeln, jedoch ist sie nicht an eine Pfarrei
gebunden. Im Folgenden wurde auf die explizite Nennung der Ortsgruppe verzichtet.
\section{Mitglieder}
Mitglied in der KjG kann jede*r werden, die*der die Grundlagen und Ziele des Verbandes bejaht.
Die*der Einzelne wird Mitglied der KjG Pfarrgemeinschaft, indem sie*er das erklärt und die Pfarrleitung diese Erklärung annimmt.
Das Mitglied ist grundsätzlich verpflichtet, den Mitgliedsbeitrag fristgerecht zu bezahlen.
Die Höhe des jeweiligen Mitgliedsbeitrags wird von der jeweiligen
Ebene festgelegt und erhoben.

Besteht keine Anbindung an eine Pfarrgemeinschaft oder Ortsgruppe, kann der*die Einzelne
die Mitgliedschaft gegenüber dem Diözesanverband erklären. Diese Erklärung wird wirksam,
wenn sie von der Diözesanleitung angenommen wird.

Eine Mitgliedschaft in der KjG kann als aktive oder passive Mitgliedschaft erworben werden.
Die Mitgliedschaft erlischt durch Austritt, Ausschluss oder Tod. Der Austritt ist für das folgende Jahr
schriftlich gegenüber der Pfarrleitung bzw. Diözesanleitung bis zum 31.12. des laufenden Jahres zu erklären.

Über den Ausschluss eines Mitglieds entscheidet die Pfarrleitung bzw. Diözesanleitung nach Anhörung der*des Betroffenen.
Das betroffene Mitglied kann gegen diesen Beschluss bei der
Mitgliederversammlung bzw. Diözesankonferenz Berufung einlegen. Diese entscheidet verbindlich.
\subsection{Aktive Mitglieder}
Als aktives Mitglied nimmt sie*er an einer oder mehreren der angebotenen Gesellungs­ und Arbeitsformen teil.

Durch die aktive Mitgliedschaft in der KjG haben Mitglieder ein Recht auf Mitbestimmung
sowie die Chance auf Aus­ und Weiterbildung. Sie können Verantwortung übernehmen und selbst
Angebote schaffen.

Jedes aktive Mitglied ist stimmberechtigt und wählbar.
\subsection{Passive Mitglieder}
Passive Mitgliedschaften in der Katholischen jungen Gemeinde dienen der ideellen und/oder finanziellen Unterstützung der Arbeit des Verbandes.
Der Mitgliedsbeitrag verbleibt bei der jeweiligen Ebene.

Die passive Mitgliedschaft schließt eine Stimmberechtigung in der Katholischen jungen
Gemeinde aus. Mitglieder einer passiven Mitgliedschaft dürfen nicht gewählt werden.

Passive Mitglieder zählen nicht in die Stimmschlüsselberechnung hinein.
\section{Die Pfarrgemeinschaft}
Die Mitglieder der Katholischen jungen Gemeinde in der Pfarrei bilden die KjG Pfarrgemeinschaft.

Sie ist Mitglied im Diözesanverband der KjG. Sie arbeitet mit anderen BDKJ-Mitgliedsverbänden
zusammen und kann mit diesen den BDKJ bilden.

Sie führt den Namen „Katholische junge Gemeinde (KjG) Pfarrgemeinschaft/Ortsgruppe N.N.“.
Das Verbandszeichen ist der Seelenbohrer.

Die KjG Pfarrgemeinschaft bestimmt nach demokratischen Regeln im Rahmen der Grundlagen
und Ziele sowie der Satzung, Leitung, Aufgaben, Gesellungs- und Arbeitsformen entsprechend
der örtlichen Situation.

Die Leiter*innen der Teams, Gruppen, Clubs oder Arbeitskreise werden entweder von den Mitgliedern
der jeweiligen Gesellungs- bzw. Arbeitsform gewählt oder durch die Pfarrleitung nach
Anhörung der Pädagogischen Leitungsrunde berufen.

Die KjG Pfarrgemeinschaft führt an den Diözesanverband einen Beitrag ab, dessen Höhe von der
Diözesankonferenz beschlossen wird. Die KjG Pfarrgemeinschaft kann einen Pfarrbeitrag erheben,
dessen Höhe in der Mitgliederversammlung beschlossen wird.

Die Vertretung im Diözesanverband erfolgt durch die Pfarrleitung oder deren Delegierte.
\subsection{Satzung der Pfarrgemeinschaft}
Die KjG Pfarrgemeinschaft kann sich im Rahmen der Grundlagen und Ziele, sowie der Satzung
des Diözesanverbandes eine eigene Pfarrsatzung geben. Existiert keine eigene Satzung, gilt die
der nächsthöheren Ebene. Diese Satzung muss mindestens enthalten:
\begin{itemize}
  \item Anerkennung und Verpflichtung auf die Grundlagen und Ziele der KjG
  \item Die Mitgliedschaft im Diözesanverband
  \item Die Zugehörigkeit zum BDKJ
\end{itemize}

Und gemäß der nachfolgenden Paragraphen
\begin{itemize}
  \item Die Mitgliederversammlung
  \item Die Pfarrleitung
  \item Den Kindersenat
\end{itemize}

Diese Satzung kann gemäß der nachfolgenden Paragraphen enthalten:
\begin{itemize}
  \item Das Orga-team
  \item Die Pädagogische Leitungsrunde
\end{itemize}

Die Satzung bedarf der Zustimmung durch die Diözesanleitung.\newline
Gegen die Entscheidung der Diözesanleitung kann beim Diözesanausschuss Einspruch erhoben
werden. Der Diözesanausschuss entscheidet abschließend.
\subsection{Ausschluss der Pfarrgemeinschaft}
Über den Ausschluss einer KjG Pfarrgemeinschaft entscheidet die Diözesanleitung nach Anhörung
der Betroffenen und der zuständigen Arbeitsgemeinschaftsleitung.
Diese Anhörung geschieht in einer außerordentlichen Mitgliederversammlung. Die betroffene KjG Pfarrgemeinschaft
kann gegen diesen Beschluss beim Diözesanausschuss Berufung einlegen. 
Der Diözesanausschuss entscheidet abschließend.

\subsection{Auflösung der Pfarrgemeinschaft}
Zu einer Auflösungsversammlung der KjG Pfarrgemeinschaft muss mindestens 14 Tage vorher
schriftlich eingeladen werden. Der Einladung ist eine ausführliche Begründung beizufügen. Der
Auflösung der KjG Pfarrgemeinschaft müssen drei Viertel der anwesenden stimmberechtigten
Mitglieder zustimmen.

Das Vermögen der KjG Pfarrgemeinschaft fällt bei Auflösung an den Diözesanverband.
Dieser ist verpflichtet, das Vermögen der KjG Pfarrgemeinschaft zweckgebunden zu verwalten.
Dies gilt im Falle eines Ausschlusses sinngemäß für Vermögen aus öffentlichen Bezuschussungen.
Sollte sich die KjG Pfarrgemeinschaft innerhalb von drei Jahren neu konstituieren, ist ihr das
Vermögen auszuhändigen. Ist dies nicht der Fall, fällt das verwaltete Vermögen an den Diözesanverband.

\section{Die Organe der KjG Pfarrgemeinschaft}
Die Organe der KjG Pfarrgemeinschaft sind die Mitgliederversammlung, die Pfarrleitung,
die Pädagogische Leitungsrunde, das Orga-Team und der Kindersenat.

\subsection{Die Mitgliederversammlung}
Die Mitgliederversammlung ist das oberste beschlussfassende Organ der KjG Pfarrgemeinschaft.
Sie trifft im Rahmen der Grundlagen und Ziele, sowie der Satzung des Diözesanverbandes und
der Beschlüsse der Diözesankonferenz die grundlegenden Entscheidungen über die Arbeit der
KjG Pfarrgemeinschaft.

\subsubsection{Aufgaben der Mitgliederversammlung}
Der Mitgliederversammlung sind insbesondere folgende Aufgaben vorbehalten:
\begin{itemize}
  \item die an die Mitgliederversammlung gerichteten Anträge
    \begin{itemize}  
      \item die Finanzen der KjG Pfarrgemeinschaft
      \item die Pfarrsatzung
      \item die Jahresplanung
      \item den Pfarrbeitrag
    \end{itemize}
  \item Entgegennahme des Jahresberichtes der Pfarrleitung
  \item Entgegennahme des Kassenberichtes
  \item Entlastung der Pfarrleitung
  \item Wahl der Pfarrleitung
  \item Wahl der Kassenprüfenden
  \item Wahl des Kindersenates (stimmberechtigt sind alle Dauermitglieder der KjG Pfarrgemeinschaft bis einschließlich 12 Jahre)
  \item Abwahl einzelner Mitglieder der Pfarrleitung
\end{itemize}

\subsubsection{Zusammensetzung der Mitgliederversammlung}
Stimmberechtigte Mitglieder der Mitgliederversammlung sind:
\begin{itemize}
  \item Die Mitglieder der KjG Pfarrgemeinschaft, sofern sie den Mitgliedsbeitrag für das laufende Jahr bezahlt haben
\end{itemize}
Beratende Mitglieder der Mitgliederversammlung sind:
\begin{itemize}
  \item Ein*e Hauptamtliche*r der Pfarrei
  \item Ein Mitglied des Sachausschuss Jugend der Pfarrei
  \item Ein Mitglied des Pfarrvorstandes des BDKJ
  \item Ein Mitglied der Leitung der zuständigen Arbeitsgemeinschaft der KjG
  \item Ein*e Vertreter*in des Diözesanverbandes der KjG
  \item Die nicht stimmberechtigten Mitglieder der KjG Pfarrgemeinschaft

\subsubsection{Einberufung, Ablauf und Beschlussfähigkeit der Mitgliederversammlung}
Die Mitgliederversammlung findet wenigstens einmal jährlich statt. Sie wird von der Pfarrleitung
drei Wochen vorher unter Bekanntgabe der Tagesordnung und der Frist für die Einreichung
der Wahlvorschläge einberufen. Jedes Mitglied wird auf geeignete Weise eingeladen.
Eine Mitgliederversammlung muss einberufen werden, wenn die Pädagogische Leitungsrunde,
der Kindersenat oder 1/5 der Mitglieder dies beantragen. Anträge können vor und während der 
Mitgliederversammlung eingebracht werden.

Anträge auf Abwahl der Pfarrleitung und Anträge auf Satzungsänderung sind den Mitgliedern
der Mitgliederversammlung mindestens 14 Tage vor dem Termin der Mitgliederversammlung mit
Begründung zuzuleiten.

Die Mitgliederversammlung beschließt und wählt mit einfacher Mehrheit der anwesenden, stimmberechtigten Mitglieder.
Stimmenthaltungen bleiben unberücksichtigt. Abstimmungen über
Abwahl der Pfarrleitung und Änderung der Satzung bedürfen der 2/3-Mehrheit der anwesenden,
stimmberechtigten Mitglieder. Für den Ablauf der Mitgliederversammlung gilt im
Übrigen die Geschäftsordnung der Diözesankonferenz sinngemäß. Über die Mitgliederversammlung
wird Protokoll geführt und den Mitgliedern zugänglich gemacht.

Die Mitgliederversammlung ist beschlussfähig, wenn frist- und formgerecht eingeladen wurde.
\end{itemize}
\subsection{Die Pfarrleitung}
\subsubsection{Aufgaben der Pfarrleitung}
Die Pfarrleitung ist verantwortlich für die Leitung und Vertretung \footnote{ vgl. §26 BGB} der KjG Pfarrgemeinschaft.
Dabei ist jedes Mitglied der Pfarrleitung alleine vertretungsberechtigt.
Ihre Aufgaben sind insbesondere:
\begin{itemize}
  \item Einberufung, Vorbereitung und Leitung der Mitgliederversammlung
  \item Einberufung, Vorbereitung und Leitung der Pädagogischen Leitungsrunde, des Orga-Teams und des Kindersenates
  \item Sorge für die Durchführung der Beschlüsse der Mitgliederversammlung
  \item Vertretung und Mitarbeit auf der Diözesanebene der KjG
  \item Vertretung der KjG Pfarrgemeinschaft in Kirche und Öffentlichkeit
  \item Zusammenarbeit mit anderen KjG Pfarrgemeinschaften
  \item Zusammenarbeit mit den anderen BDKJ Mitgliedsverbänden
  \item Zusammenarbeit mit den in den Pfarreien tätigen Gemeinschaften und Gremien
  \item Verantwortung für die Finanzen
  \item Sorge um die Aus- und Weiterbildung der Mitarbeitenden durch den Verband (insbesondere der Gruppenleiter*innen)
  \item Einbringen von Spiritualität
\end{itemize}

\subsubsection{Zusammensetzung der Pfarrleitung}

Die Pfarrleitung ist {\color{red} geschlechtergerecht\footnoteremember{geschlechtergerecht}{Geschlechtergerecht im Rahmen dieser Satzung bedeutet: Gremien (und Ämter) werden mit männlichen und
weiblichen Personen paritätisch besetzt. Bei Gremien mit einer Größe von bis zu 10 Personen wird zusätzlich
eine, bei mehr als 10 Personen zwei Stellen für Personen diversen Geschlechts eingerichtet.}} zu besetzen, ihr gehören mindestens an:
Stimmberechtigt:
\begin{itemize}
  \item 5 Pfarrleiter*innen (2m, 2w, 1d)
  \item 2 Geistliche Leiter*innen unterschiedlichen Geschlechts\footnoteremember{Berechtigung Geist}{
    Das Amt der Geistlichen Leiterin und des Geistlichen Leiters kann von Personen wahrgenommenwerden,
    die eine theologische oder religionspäd. Ausbildung abg. haben.
  }
  \item {\color{red} 1 Geistlicher Leiter \footnoterecall{Berechtigung Geist}}
  \item {\color{red} 1 Geistliche Leiterin \footnoterecall{Berechtigung Geist}}
\end{itemize}
Die Aufgaben der Pfarrleitung können auch wahrgenommen werden, wenn nicht alle Stellen besetzt sind.
{\color{red} Von der Verpflichtung zur geschlechtergerechten Besetzung sind die KjG Pfarrgemeinschaften ausgenommen, in
denen nur Personen eines Geschlechts vertreten sind.}

Mindestens ein Mitglied der Pfarrleitung muss voll geschäftsfähig sein.
Die stimmberechtigten Mitglieder der Pfarrleitung werden von der Mitgliederversammlung für
zwei Jahre gewählt. Die stimmberechtigten Mitglieder der Pfarrleitung können ihren Rücktritt nur
gegenüber der Mitgliederversammlung erklären.

Sind alle Stellen der Pfarrleitung vakant, so dürfen deren Aufgaben von der
Diözesanleitung übernommen werden. In diesem Fall hat die Diözesanleitung die Möglichkeit eine
Stimme bei der Mitgliederversammlung wahrzunehmen.

\subsection{Das Orga-Team}

\subsubsection{Aufgaben des Orga-Teams}

Das Orga-Team berät im Rahmen der Beschlüsse der Mitgliederversammlung die Arbeit der KjG
Pfarrgemeinschaft und stimmt die Interessen der einzelnen Gesellungs- und Arbeitsformen ab,
unbenommen der Letztverantwortung der Pfarrleitung.

Dem Orga-Team sind insbesondere folgende Aufgaben vorbehalten:
\begin{itemize}
  \item Planung und Sorge für die Durchführung der Veranstaltungen und Aktionen der KjG Pfarrgemeinschaft
  \item Gewinnung von Leiter*innen und freien Mitarbeitenden
\end{itemize}

\subsubsection{Zusammensetzung und Einberufung des Orga-Teams}
Mitglied des Orga-Teams kann jedes Mitglied der Pfarrgemeinschaft werden. Das Orga-Team
trifft sich nach Bedarf und wird von der Pfarrleitung einberufen und geleitet.

\subsection{Die Pädagogische Leitungsrunde}

\subsubsection{Aufgaben der Pädagogischen Leitungsrunde}
Die Pädagogische Leitungsrunde dient den Leiter*innen der einzelnen Gesellungs- und Arbeitsformen als Ort für:
\begin{itemize}
  \item Erfahrungsaustausch
  \item Weiterbildung
  \item Informationen über die Situation der {\color{red} Kinder} in der Pfarrgemeinde
  \item Reflexion der Gruppenarbeit und des eigenen Leitungsverhaltens
\end{itemize}

\subsubsection{Zusammensetzung und Einberufung der Pädagogischen Leitungsrunde}
Zur pädagogischen Leitungsrunde gehören:
\begin{itemize}
  \item Die Pfarrleitung
  \item Die Leiter*innen der einzelnen Gesellungs- und Arbeitsformen
\end{itemize}

Gäste können von der Pädagogischen Leitungsrunde eingeladen werden.

Die Pädagogische Leitungsrunde wird regelmäßig, mindestens viermal im Jahr,
von der Pfarrleitung einberufen und geleitet.

\subsection{Der Kindersenat}
Der Kindersenat dient der Kindermitbestimmung in der Zeit zwischen den Mitgliederversammlungen.
In den Kindersenat können Dauermitglieder der KjG Pfarrgemeinschaft bis einschließlich 12
Jahre gewählt werden.

Die stimmberechtigten Mitglieder des Kindersenats werden auf der Mitgliederversammlung von
den bis einschließlich 12 Jahre alten Dauermitgliedern für die Dauer von einem Jahr gewählt.

\subsubsection{Aufgaben des Kindersenates}
Zu den Aufgaben des Kindersenats gehören:
\begin{itemize}
  \item Anliegen von Kindern in Pfarrleitungs- und Pädagogischer Leitungsrunde einbringen
  \item Beratende Funktion bei Aktionen und Veranstaltungen für Kinder in der KjG Pfarrgemeinschaft
\end{itemize}

\subsubsection{Zusammensetzung und Einberufung des Kindersenats}

Der Kindersenat ist {\color{red}geschlechtergerecht} zu besetzen, ihm gehören mindestens an:
Stimmberechtigt:
{\color{red}
\begin{itemize}
  \item 2 männliche Kinder
  \item 2 weibliche Kinder
  \item 1 diverses Kind
\end{itemize}
} %end color red
Die Aufgaben des Kindersenates können auch dann wahrgenommen werden, wenn nicht alle Ämter besetzt sind.
{\color{red}Von der Verpflichtung zur geschlechtergerechten Besetzung sind die KjG Pfarrgemeinschaften ausgenommen,
in denen nur Personen eines Geschlechts vertreten sind.}

Der Kindersenat wird regelmäßig, mindestens zweimal im Jahr, von der Pfarrleitung einberufen
und von einem Mitglied der Pfarrleitung geleitet.

\chapter{KjG auf mittlerer Ebene}

\section{KjG Arbeitsgemeinschaften}

Die KjG Pfarrgemeinschaften des Diözesanverbandes können zur besseren Wahrnehmung ihrer
Aufgaben auf der mittleren Ebene Arbeitsgemeinschaften bilden.

Sie führt den Namen „Katholische junge Gemeinde (KjG) Arbeitsgemeinschaft N.N.“.
Das Vebandszeichen ist der Seelenbohrer.

Vordringliche Aufgabe der Arbeitsgemeinschaft ist die Unterstützung, Förderung und Koordinierung
der Arbeit der KjG Pfarrgemeinschaften.

Die Arbeitsgemeinschaft hat keine Beitragshoheit.

Alle beteiligten KjG Pfarrgemeinschaften müssen der Arbeitsgemeinschaft im Rahmen der Grundlagen
und Ziele der KjG, sowie der Satzung der KjG Diözesanverband Regensburg eine eigene
Satzung geben. Die Satzungsgebung muss einstimmig auf einer Konferenz der beteiligten Pfarreien
beschlossen werden. Satzungsänderungen sind dann mit einer 2/3 Mehrheit möglich.

Die Satzung muss enthalten:
\begin{itemize}
  \item Anerkennung und Verpflichtung auf die Grundlagen und Ziele der KjG
  \item Die Mitgliedschaft im KjG Diözesanverband Regensburg
  \item Die Zugehörigkeit zum BDKJ
  \item Eine mindestens jährlich stattfindende Konferenz der beteiligten Pfarrgemeinschaften, bei der
        die Geschäftsordnung der Diözesankonferenz der KjG Diözesanverband Regensburg gilt
  {\color{red}\item Die Wahl einer geschlechtergerecht zu besetzenden Leitung}
\end{itemize}

Die Satzung bedarf der Zustimmung der Diözesanleitung. Gegen die Entscheidung der Diözesanleitung
kann beim Diözesanausschuss Einspruch erhoben werden. Der Diözesanausschuss entscheidet
abschließend. Zur Satzungsgebung ist die Diözesanleitung
anzuhören.

Der Arbeitsgemeinschaftskonferenz sind insbesondere folgende Aufgaben vorbehalten:
\begin{itemize}
  \item Erfahrungsaustausch und Koordinierung der Arbeit der beteiligten KjG Pfarrgemeinschaften
  \item Beratung der Arbeit des Diözesanverbandes
  \item Beratung und Beschlussfassung über Veranstaltungen und Aktionen der Arbeitsgemeinschaft
  \item Planung von Schulungen für die Verantwortlichen der KjG Pfarrgemeinschaften
  \item Beratung und Beschlussfassung über die Finanzen der Arbeitsgemeinschaft
  \item Entgegennahme des Berichtes der Arbeitsgemeinschaftsleitung
\end{itemize}
\section{KjG Bezirksverbände}
Der Diözesanverband kann sich in Bezirksverbände gliedern. Dafür gelten die entsprechenden
Bestimmungen der Satzung des Bundesverbandes.
\chapter{KjG in der Diözese}
Der Diözesanverband der Katholischen jungen Gemeinde ist der Zusammenschluss der
KjG Pfarrgemeinschaften in der Diözese.

Der Diözesanverband ist Mitglied im Bundesverband der Katholischen jungen Gemeinde und im
Diözesanverband des BDKJ.

Er führt den Namen „Katholische junge Gemeinde (KjG) Diözesanverband Regensburg“,
auch kurz KjG Diözesanverband Regensburg, mit Sitz in Regensburg.
Das Verbandszeichen ist der Seelenbohrer.
Der Diözesanverband ist ein nicht rechtsfähiger Verein.

Aufgabe des Diözesanverbandes ist die Unterstützung, Förderung und Koordinierung der Arbeit
der KjG Pfarrgemeinschaften und der Arbeitsgemeinschaften der KjG Pfarrgemeinschaften und
deren Vertretung in Kirche und Gesellschaft.

\section{Gemeinnützigkeit}

\subsection{}
Der KjG Diözesanverband Regensburg verfolgt ausschließlich und unmittelbar gemeinnützige
und kirchliche Zwecke im Sinne des Abschnitts ,,Steuerbegünstigte Zwecke`` der Abgabenordnung.

\subsection{}
\label{sec:Zweck}
Zweck des KjG Diözesanverbandes Regensburg ist die
Förderung der Religion \footnote{§52 Abs. 2 S.1 Nr.2 AO},
der Jugendhilfe \footnote{§52 Abs. 2 S 1 Nr.4 AO},
der Erziehung, Volks- und Berufsbildung einschließlich
der Studentenhilfe \footnote{§52 Abs. 2 S 1 Nr.7 AO}, der internationalen Gesinnung und des
Völkerverständigungsgedankens \footnote{§52 Abs. 2 S.1 Nr. 13 AO}, des bürgerlichen Engagements zugunsten
gemeinnütziger und kirchlicher Zwecke\footnote{§ 52 Abs 2 S. 1 Nr.25 AO}
sowie die Verfolgung kirchlicher Zwecke \footnote{§54 AO}

\subsection{}
Der Satzungszweck wird verwirklicht insbesondere durch:
\begin{itemize}
  \item die Wahrnehmung kirchlicher Kinder- und Jugendarbeit insbesondere in der Diözese Regensburg
        in Einheit mit der Gesamtkirche und in Übereinstimmung mit den Grundrechten selbst,
  \item die Schaffung von Begegnungsmöglichkeiten im Rahmen der Organisation oder Durchführung von
        Begegnungs- und Bildungsmaßnahmen sowie Aktionen
  \item die Förderung demokratischen, gleichberechtigten und solidarischen Engagements, das sich
        gegen jede Art von Ausgrenzung oder Unterdrückung von Menschen wendet,
  \item die Förderung einer ökologisch verantworteten Lebensweise um die Zerstörung der
          natürlichen Lebensgrundlage einzudämmen
  \item die nationale und internationale Zusammenarbeit um partnerschaftlich und solidarisch für
        eine weltweite Etablierung von gleichen und gerechten Lebensbedingungen für alle Menschen einzustehen
  \item die Schaffung von Raum für Kinder und Jugendliche sowie junge Erwachsene und deren
        Gruppierungen:
    \begin{itemize}
      \item um Begegnungen und Beziehungen zu fördern und durch gemeinsame Erlebnisse und gemeinsames Handeln
            das Zugehörigkeitsgefühl und die Glaubensgemeinschaft zu stärken
      \item zur ständigen Wertorientierung und Wertschätzung innerhalb der Gruppierung und der Kirche
      \item zur Standortüberprüfung und Entwicklung von Lebensperspektiven in Einheit mit einem 
            selbstverantworteten religiösen Lebens
      \item zur Ermutigung soziale, politische und pädagogische Verantwortung zu übernehmen und 	
            persönliche Interessen und Fähigkeiten zu entwickeln
      \item zur Schaffung von Impulsen und Möglichkeiten zur Entwicklung eines demokratischen Zusammenwirkens
            und Handelns in Einheit mit der Gesamtkirche und in Übereinstimmung 
            mit den Grundrechten in einer globalisierten Welt.
    \end{itemize}
\end{itemize}
\subsection{}
Der KjG Diözesanverband Regensburg darf seinen Satzungszweck auch durch Hilfspersonen \footnote{§57 Abs. 1 S. 2 AO}
verwirklichen.
\subsection{}
Der KjG Diözesanverband Regensburg ist selbstlos tätig; er verfolgt nicht in erster Linie eigenwirtschaftliche Zwecke.
\subsection{}
Mittel des KjG Diözesanverbandes Regensburg dürfen nur für die satzungsmäßigen Zwecke
verwendet werden. Die Mitglieder erhalten keine Zuwendungen aus Mitteln der Körperschaft.
\subsection{}
Es darf keine Person durch Ausgaben, die dem Zweck des KjG Diözesanverbandes Regensburg
fremd sind, oder durch unverhältnismäßig hohe Vergütungen begünstigt werden.
\section{Die Organe des Diözesanverbandes}
Die Organe des KjG Diözesanverbandes Regensburg sind die Diözesankonferenz, der Diözesanausschuss und die
Diözesanleitung.

\subsection{Die Diözesankonferenz}
Die Diözesankonferenz ist das oberste beschlussfassende Organ des KjG Diözesanverbandes Regensburg. Sie
trifft im Rahmen der Grundlagen und Ziele, sowie der Satzung des Bundesverbandes und der
Beschlüsse der Bundeskonferenz die grundlegenden Entscheidungen über die Arbeit des Diözesanverbandes.
\subsubsection{Aufgaben der Diözesankonferenz}
Der Diözesankonferenz sind insbesondere folgende Aufgaben vorbehalten:
\begin{itemize}	
  \item Beschlussfassung über:
    \begin{itemize} 
      \item die Diözesansatzung
      \item den Diözesanbeitrag
      \item die Jahresplanung
      \item das Schulungsprogramm
      \item gemeinsame Aktionen
      \item die Einrichtung und Auflösung von diözesanen Teams und Arbeitsgruppen
    \end{itemize}
  \item Entgegennahme der Tätigkeitsberichte der Diözesanleitung und des Diözesanausschusses
  \item Entgegennahme des Finanzberichtes
  \item Entlastung der Diözesanleitung
  \item Wahl:
    \begin{itemize}
      \item der Diözesanleitung
      \item des Diözesanausschusses
      \item des Wahlausschusses
      \item der Kassenprüfung
      \item der Delegierten für die Bundeskonferenz, den Bundesrat und die 
            Mitgliederversammlungen der Bundesstelle der Katholischen jungen Gemeinde e.V. und sowie für die
            Diözesanversammlung des BDKJ
      \item der Delegierten der Mitgliederversammlung der KJG Landesstelle e.V., sofern die Diözesanleitung
            unbesetzt ist
      \item ggf. der Mitglieder von Sachausschüssen
    \end{itemize}
  \item Abwahl einzelner Mitglieder der Diözesanleitung beziehungsweise des Diözesanausschusses
\end{itemize}
\subsubsection{Ausschüsse}
Die Diözesankonferenz kann für bestimmte Aufgaben {\color{red}geschlechtergerecht} besetzte Sachausschüsse einrichten.
Sachausschüsse zu geschlechtsspezifischen Belangen sind hiervon ausgenommen.
Den Vorsitz der Sachausschüsse hat ein Mitglied der Diözesanleitung inne, dieser kann delegiert
werden.

Der Wahlausschuss leitet die Wahlen. Der Wahlausschuss ist {\color{red}geschlechtergerecht} zu besetzen. Den Vorsitz
des Wahlausschusses hat ein Mitglied der Diözesanleitung inne, dieser kann delegiert werden.

\subsubsection{Zusammensetzung der Diözesankonferenz}
Stimmberechtigte Mitglieder der Diözesankonferenz sind:
\begin{itemize}
  \item 2 Delegierte pro KjG Pfarrgemeinschaft
  \item Die Mitglieder der Diözesanleitung
\end{itemize}

Die Delegation ist folgendermaßen zu besetzen:
\begin{itemize}
  \item{\color{red} 2 Mitglieder der Pfarrleitung bzw. von Pfarrleitung oder Mitgliederversammlung
        Delegierte unterschiedlichen Geschlechts}
\end{itemize}

{\color{red}Von der Verpflichtung zur geschlechtergerechten Besetzung sind die KjG Pfarrgemeinschaften ausgenommen,
in denen nur Personen eines Geschlechts vertreten sind.}

Beratende Mitglieder sind:
\begin{itemize}
  \item Die Diözesanreferent*innen
  \item Die Mitglieder des Diözesanausschusses
  \item Ein Mitglied von Sachausschüssen und diözesanen Projektgruppen
  \item Ein Mitglied der Bundesleitung der Katholischen jungen Gemeinde
  \item Ein*e Vertreter*in des Landesvorstandes der KjG-Landesarbeitsgemeinschaft Bayern
  \item Ein Mitglied des BDKJ Diözesanvorstandes
  \item Der*die Vorsitzende des Vereins zur Förderung der Katholischen jungen Gemeinde in der
        Diözese Regensburg e.V.
  \item Je ein Mitglied der diözesanen Teams und Arbeitsgruppen
        \footnoteremember{Dauermitglied}{Das jeweilige Mitglied muss Dauermitglied im KjG Diözesanverband Regensburg sein}
  \item Je ein Mitglied der Leitung der Arbeitsgemeinschaften der Pfarreien \footnoterecall{Dauermitglied}
\end{itemize}

Gäste können von der Diözesanleitung eingeladen werden.
\subsubsection{Einberufung und Ablauf der Diözesankonferenz}
Die Diözesankonferenz tritt mindestens einmal jährlich zusammen und wird von der Diözesanleitung
einberufen und geleitet. Sie ist in der Regel öffentlich.
Eine außerordentliche Diözesankonferenz muss einberufen werden, wenn der Diözesanausschuss oder ein
Drittel der Pfarrgemeinschaften dies beantragen.

Der Ablauf der Diözesankonferenz regelt sich nach der Geschäftsordnung.

\subsubsection{Änderung der Satzung des Diözesanverbandes}
Änderungen der Diözesansatzung können nur beschlossen werden, wenn zwei Drittel der anwesenden
stimmberechtigten Mitglieder zustimmen und der Änderungsantrag den Mitgliedern der
Diözesankonferenz mindestens vier Wochen vorher schriftlich mitgeteilt worden ist.

\subsection{Der Diözesanausschuss}
Der Diözesanausschuss berät im Rahmen der Grundlagen und Ziele und der Beschlüsse der Diözesankonferenz
über die Arbeit und beschließt über laufende wichtige Angelegenheiten des KjG Diözesanverbandes Regensburg.

\subsubsection{Aufgaben des Diözesanausschusses}
Dem Diözesanausschuss sind insbesondere folgende Aufgaben vorbehalten:
\begin{itemize}
  \item Planung und Vorbereitung der Diözesankonferenz
  \item Sorge für die Durchführung der Beschlüsse der Diözesankonferenz
  \item Beschlussfassung über den Etat des Diözesanverbandes
  \item Schlichtung und Entscheidung bei Konfliktfällen
        \footnote{Betroffene Mitglieder haben bei der Entscheidung kein Stimmrecht}
\end{itemize}

\subsubsection{Zusammensetzung des Diözesanauschusses}
Stimmberechtigte Mitglieder des Diözesanausschusses sind:
\begin{itemize}
  \item 4 weibliche Mitglieder der Pfarrleitungen bzw. Mitglieder einer Pfarrgemeinschaft, die von der
        Mitgliederversammlung ein Mandat erhalten haben. Von diesen sollte mindestens eine Person
        Geistliche Leiterin sein.
  \item 4 männliche Mitglieder der Pfarrleitungen bzw. Mitglieder einer Pfarrgemeinschaft, die von
        der Mitgliederversammlung ein Mandat erhalten haben. Von diesen sollte mindestens eine
        Person Geistlicher Leiter sein.
  {\color{red}\item 1 diverses Mitglied der Pfarrleitungen bzw. Mitglieder einer Pfarrgemeinschaft, die von
        der Mitgliederversammlung ein Mandat erhalten haben.}
  \item Die Mitglieder der Diözesanleitung
\end{itemize}

Beratende Mitglieder sind:
\begin{itemize}
  \item Die Diözesanreferent*innen
\end{itemize}

Die Aufgaben des Diözesanausschusses können auch dann wahrgenommen werden, wenn nicht
alle Stellen besetzt sind.

Das Mindestalter für den Diözesanausschuss liegt bei 16 Jahren. Von den stimmberechtigten Mitgliedern
des Diözesanausschusses, die nicht Teil der Diözesanleitung sind, muss aber mindestens
ein Mitglied, unabhängig des Geschlechts, voll geschäftsfähig sein.

Gäste können von der Diözesanleitung oder dem Diözesanausschuss eingeladen werden.

Die Vertretungen der Pfarrgemeinschaften werden von der Diözesankonferenz für zwei Jahre
gewählt. Die Wahl ist persönlich; eine Vertretung im Diözesanausschuss ist nicht möglich. Mit
dem Wegfall der Voraussetzung für den Diözesanausschuss erlischt die Mitgliedschaft im Diözesanausschuss.

\subsubsection{Einberufung und Ablauf des Diözesanausschusses}
Der Diözesanausschuss tritt nach Bedarf, mindestens jedoch zweimal jährlich zusammen. Er wird
von der Diözesanleitung mindestens 14 Tage vorher einberufen. Den Vorsitz hat die Diözesanleitung.

\subsection{Die Diözesanleitung}
Die Diözesanleitung führt die Geschäfte und vertritt den KjG Diözesanverband Regensburg in sämtlichen
Angelegenheiten gerichtlich und außergerichtlich.
\subsubsection{Aufgaben der Diözesanleitung}
Zu den Aufgaben der Diözesanleitung gehören insbesondere:
\begin{itemize}
  \item Leitung und Geschäftsführung des Diözesanverbandes im Rahmen der Grundlagen und Ziele
        des Verbandes, sowie der Satzung und der Beschlüsse der Organe des Bundes- und Diözesanverbandes
  \item Vertretung des Diözesanverbandes im Bundesverband
  \item Vertretung des Diözesanverbandes im BDKJ auf Diözesanebene
  \item Vertretung des Diözesanverbandes in der Landesarbeitsgemeinschaft der KjG
  \item Vertretung des Diözesanverbandes in Kirche und Gesellschaft
\end{itemize}

Zur Erfüllung ihrer Aufgaben kann die Diözesanleitung mit Zustimmung des Diözesanausschusses
Referent*innen, Sachbearbeitende sowie Mitarbeitende berufen.

\subsubsection{Zusammensetzung der Diözesanleitung}
Zur Diözesanleitung gehören:
\begin{itemize}
  {\color{red}\item 3 weibliche Mitglieder}, wovon eine Geistliche Leiterin 
       \footnote{
           Das Amt der Geistlichen Leiterin kann von Frauen wahrgenommen werden, die eine theologische
           oder religionspädagogische Ausbildung abgeschlossen haben.
       }
       ist
  {\color{red}\item 3 männliche Mitglieder}, wovon einer Geistlicher Leiter
        \footnote{
            Das Amt des Geistlichen Leiters kann von Männern wahrgenommen werden, die eine theologische
            oder religionspädagogische Ausbildung abgeschlossen haben. Derzeit kann dieses Amt in Absprache
            mit dem bischöflichen Stuhl nur von ordinierten, katholischen Priestern wahrgenommen werden.	
        }
        ist
    {\color{red}\item 1 diverses Mitglied}
\end{itemize}

Die Aufgaben der Diözesanleitung können auch dann wahrgenommen werden, wenn nicht alle
Ämter besetzt sind. Mindestens ein Mitglied der Diözesanleitung muss voll geschäftsfähig sein.

Kann eine Stelle der Geistlichen Leitung nicht besetzt werden, kann eine weitere Diözesanleitung
gewählt werden. Kann keine der beiden Geistlichen Leitungsstellen besetzt werden, entscheidet
die Diözesankonferenz, welche Position bis zur nächsten Wahl unbesetzt bleibt.

Die Diözesanleitung wird von der Diözesankonferenz für zwei Jahre gewählt. Die Mitglieder der
Diözesanleitung können ihren Rücktritt nur vor der Diözesankonferenz erklären.

Den Mitgliedern der Diözesanleitung werden die bei der Verbandsarbeit entstandenen, angemessenen Auslagen
ersetzt. Mitglieder der Diözesanleitung können darüber hinaus eine angemessene Vergütung erhalten.
Die Vergütung für den Zeitaufwand bedarf dem Grunde und der Höhe nach der vorherigen Beschlussfassung
des Diözesanausschusses.

\subsubsection{Kontakt zu KjG Pfarrgemeinschaften}
Die Wahrnehmung der Kontakte zu den KjG Pfarrgemeinschaften ist Aufgabe von Diözesanleitung
und gewähltem Diözesanausschuss. Bei Bedarf können weitere interessierte KjG Mitglieder,
vorzugsweise mit Erfahrung in der KjG Pfarreiarbeit, mit dieser Aufgabe betraut werden.

\section{Auflösung des Diözesanverbandes}
Zu einer Auflösungsversammlung des KjG Diözesanverbandes Regensburg muss mindestens 28 Tage vorher
schriftlich eingeladen werden. Der Einladung ist eine Begründung hinzuzufügen. Drei Viertel der
anwesenden stimmberechtigten Mitglieder müssen der Auflösung zustimmen. Das weitere Vorgehen
im Falle der Auflösung regelt die Satzung des Bundesverbandes.

Bei Auflösung oder Aufhebung des KjG Diözesanverbandes Regensburg oder bei Wegfall steuerbegünstigter
Zwecke fällt das Vermögen des Diözesanverbandes an den KjG Bundesverband, der es unmittelbar und
ausschließlich für gemeinnützige oder kirchliche Zwecke im Sinne der \ref{sec:Zweck} zu verwenden hat.
\chapter{Schlussbestimmungen}
Die Neufassung der Satzung tritt nach ihrer Beschlussfassung durch die Bundeskonferenz der
Katholischen jungen Gemeinde 2017 in Kraft.

% Ende des Dokumentes
\end{flushleft}
\end{document}
