\documentclass[12pt]{report}
\usepackage[a4paper, left=2cm,top=2cm]{geometry}

%keine einzelnen Zeilen am Anfang/Ende:
\widowpenalty = 10000
\clubpenalty = 10000
%------------------------------------
%   Required Packages
%------------------------------------
\usepackage[table,xcdraw]{xcolor} %table, xcdraw are needed for table colors
\usepackage{enumitem} %needed to style tables
\usepackage{float} %needed for positioning tables and text in correct order
\usepackage{graphicx}
\usepackage[pages=some]{background}
\usepackage[utf8]{inputenc}
\usepackage[english,ngerman]{babel} % deutsche Bezeichnungen und Worttrennung… 
\usepackage[T1]{fontenc} % Wird u.a. f\"ur das Trennen von W\"ortern  mit Umlauten genutzt.
\usepackage{tgbonum} %schriftarten
%------------------------------------------
%    Farben
%------------------------------------------

\definecolor{weiß}{RGB}{255,255,255}
\definecolor{kjggrau}{RGB}{119,120,123}
\definecolor{kjghellgrau}{RGB}{192,192,192}
\definecolor{kjgblau}{RGB}{54,95,145}
\definecolor{kjgtuerkis}{RGB}{0, 181,190}

%------------------------------------------
% Start-cover als Einband 
%------------------------------------------
\backgroundsetup{
 scale=1,
 angle=0,
 color=white,
 placement=top,
 opacity = 1,
 contents= {
  \includegraphics[width=\paperwidth, height=26.80cm]{satzung_einband.png}
 }
}

%--------------------------------------
% Fußnoten mehfrach referenzieren:
% Quelle:
% https://www.macuser.de/threads/latex-fussnote-mehrfach-referenzieren.519258/
\newcommand{\footnoteremember}[2]{
  \footnote{#2}
  \newcounter{#1}
  \setcounter{#1}{\value{footnote}}
}
\newcommand{\footnoterecall}[1]{%
  \footnotemark[\value{#1}]
}
%--------------------------------------

%--------------------------------------
% Add a counter to number the tables for the changes
% Source:
% https://tex.stackexchange.com/questions/503668/increment-a-counter
\newcounter{tablecounter}
%Define a counter to automatically increase and display the counter
\newcommand\showcounter{\addtocounter{tablecounter}{1}\thetablecounter}
%--------------------------------------


\title{Satzung KjG Regensburg}
\author{}
\setcounter{chapter}{-1}

\begin{document}
\setcounter{page}{1}


%alles links zentrieren und keine einrückung bei Paragraphen:
\begin{flushleft}
%Jetzt geht es mit Inhalt los:
Die Diözesankonferenz möge beschließen die Satzung des KjG Diözesanverbandes wie folgt anzupassen:


\subsubsection{Aufgaben der Mitgliederversammlung}
Der Mitgliederversammlung sind insbesondere folgende Aufgaben vorbehalten:
\begin{itemize}
  \item die an die Mitgliederversammlung gerichteten Anträge
    \begin{itemize}  
      \item die Finanzen der KjG Pfarrgemeinschaft
      \item die Pfarrsatzung
      \item die Jahresplanung
      \item den Pfarrbeitrag
    \end{itemize}
  \item Entgegennahme des Jahresberichtes der Pfarrleitung
  \item Entgegennahme des Kassenberichtes
  \item Entlastung der Pfarrleitung
  \item Wahl der Pfarrleitung
  \item Wahl der Kassenprüfenden
\end{itemize}
\begin{table}[H]
	\begin{tabular}{|l|}
		\hline
		\rowcolor[HTML]{FFCC67} 
		\rule[-1ex]{0pt}{4ex} \textbf{KjG Regensburg (11/2019)}     \hspace{0.6\textwidth} \showcounter        \\ \hline
		\rule[-1ex]{0pt}{4ex} \begin{minipage}[t]{\textwidth} 
\begin{itemize}
  \item Wahl des Kindersenates (stimmberechtigt sind alle Dauermitglieder der KjG Pfarrgemeinschaft bis einschließlich 12 Jahre)
\end{itemize}
			\rule[-1.2ex]{0pt}{0pt}
		\end{minipage}

		\\ \hline
		\rowcolor[HTML]{9AFF99} 
		\rule[-1ex]{0pt}{4ex}\begin{minipage}[t]{\textwidth}
      \textbf{
      Vorschlag: Wir wollen die Erklärung entfernen, da das  bei "1.3.5 Der Kindersenat" bereits definiert ist.
      Es wäre verwirrend, wenn diese Erklärung in Zukunft einmal von der tatsächlichen Regelung in der Satzung abweichen sollte.\\}  
		\end{minipage}             \\ \hline
		\rule[-1ex]{0pt}{4ex}\begin{minipage}[t]{\textwidth}
\begin{itemize}
  \item Wahl des Kindersenates
\end{itemize}
		\end{minipage}
		\\ \hline
	\end{tabular}
\end{table}
\begin{itemize}
  \item Wahl des Kindersenates (stimmberechtigt sind alle Dauermitglieder der KjG Pfarrgemeinschaft bis einschließlich 12 Jahre)
  \item Abwahl einzelner Mitglieder der Pfarrleitung
\end{itemize}

\subsubsection{Zusammensetzung der Pfarrleitung}

Die Pfarrleitung ist paritätisch zu besetzen, ihr gehören mindestens an:
Stimmberechtigt:
\begin{itemize}
  \item 2 Pfarrleiter
  \item 2 Pfarrleiterinnen
  \item 1 Geistlicher Leiter \footnoteremember{Berechtigung Geist}{
    Das Amt der Geistlichen Leiterin und des Geistlichen Leiters kann von Personen wahrgenommenwerden,
    die eine theologische oder religionspäd. Ausbildung abg. haben.
  }
  \item 1 Geistliche Leiterin \footnoterecall{Berechtigung Geist}
\end{itemize}
Die Aufgaben der Pfarrleitung können auch wahrgenommen werden, wenn nicht alle Stellen besetzt sind.
Von der Verpflichtung zur Parität sind die KjG Pfarrgemeinschaften ausgenommen, in
denen nur Mädchen und Frauen oder Jungen und Männer vertreten sind.

Mindestens ein Mitglied der Pfarrleitung muss voll geschäftsfähig sein.
\begin{table}[H]
	\begin{tabular}{|l|}
		\hline
		\rowcolor[HTML]{FFCC67} 
		\rule[-1ex]{0pt}{4ex} \textbf{KjG Regensburg (11/2019)}     \hspace{0.6\textwidth} \showcounter        \\ \hline
		\rule[-1ex]{0pt}{4ex} \begin{minipage}[t]{\textwidth} 
			---
			\rule[-1.2ex]{0pt}{0pt}
		\end{minipage}
		
		\\ \hline
		\rowcolor[HTML]{CBCEFB} 
		\rule[-1ex]{0pt}{4ex}\textbf{Bundessatzung (10/2019)} \\ \hline
		\rule[-1ex]{0pt}{4ex}\begin{minipage}[t]{\textwidth} 
			Die Pfarrleitung kann für die Kassenführung eine*n Kassierer*in berufen.
			\rule[-1.2ex]{0pt}{0pt}
		\end{minipage}
		\\ \hline
		\rowcolor[HTML]{9AFF99} 
		\rule[-1ex]{0pt}{4ex}\begin{minipage}[t]{\textwidth}
			\textbf{Vorschlag: Wir schlagen die Bundessatzung mit einem Zusatz vor, um klarzumachen, dass es sich um eine Aufgabendelegation handelt und damit keine Stimme in der Pfarrleitung verbunden ist.\\}  
		\end{minipage}             \\ \hline
		\rule[-1ex]{0pt}{4ex}\begin{minipage}[t]{\textwidth}
			Die Pfarrleitung kann für die Kassenführung eine*n Kassierer*in berufen. Mit diesem Amt sind keine zusätzlichen Stimmrechte verbunden.\\
		\end{minipage}
		\\ \hline
	\end{tabular}
\end{table}

\begin{table}[H]
	\begin{tabular}{|l|}
		\hline
		\rowcolor[HTML]{FFCC67} 
		\rule[-1ex]{0pt}{4ex} \textbf{KjG Regensburg (11/2019)}     \hspace{0.6\textwidth} \showcounter        \\ \hline
		\rule[-1ex]{0pt}{4ex} \begin{minipage}[t]{\textwidth} 
			Die stimmberechtigten Mitglieder der Pfarrleitung werden von der Mitgliederversammlung für
			zwei Jahre gewählt.
			\rule[-1.2ex]{0pt}{0pt}
		\end{minipage}
		
		\\ \hline
		\rowcolor[HTML]{CBCEFB} 
		\rule[-1ex]{0pt}{4ex}\textbf{Bundessatzung (10/2019)} \\ \hline
		\rule[-1ex]{0pt}{4ex}\begin{minipage}[t]{\textwidth} 
			Die Mitglieder der Pfarrleitung werden von der Mitgliederversammlung für zwei Jahre gewählt.
			\rule[-1.2ex]{0pt}{0pt}
		\end{minipage}
		\\ \hline
		\rowcolor[HTML]{9AFF99} 
		\rule[-1ex]{0pt}{4ex}\begin{minipage}[t]{\textwidth}
			\textbf{Vorschlag: Wir schlagen die Bundessatzung vor, da alle Mitglieder der Pfarrleitung ein Stimmrecht besitzen.\\}  
		\end{minipage}             \\ \hline
		\rule[-1ex]{0pt}{4ex}\begin{minipage}[t]{\textwidth} 
			Die Mitglieder der Pfarrleitung werden von der Mitgliederversammlung für zwei Jahre gewählt.\\
		\end{minipage}
		\\ \hline
	\end{tabular}
\end{table}

\begin{table}[H]
	\begin{tabular}{|l|}
		\hline
		\rowcolor[HTML]{FFCC67} 
		\rule[-1ex]{0pt}{4ex} \textbf{KjG Regensburg (11/2019)}     \hspace{0.6\textwidth} \showcounter        \\ \hline
		\rule[-1ex]{0pt}{4ex} \begin{minipage}[t]{\textwidth} 
			Die stimmberechtigten Mitglieder der Pfarrleitung können ihren Rücktritt nur
			gegenüber der Mitgliederversammlung erklären.
			\rule[-1.2ex]{0pt}{0pt}
		\end{minipage}
		
		\\ \hline
		\rowcolor[HTML]{CBCEFB} 
		\rule[-1ex]{0pt}{4ex}\textbf{Bundessatzung (10/2019)} \\ \hline
		\rule[-1ex]{0pt}{4ex}\begin{minipage}[t]{\textwidth} 
			Die Mitglieder der Pfarrleitung können ihren Rücktritt nur gegenüber der Mitgliederversammlung erklären.
			\rule[-1.2ex]{0pt}{0pt}
		\end{minipage}
		\\ \hline
		\rowcolor[HTML]{9AFF99} 
		\rule[-1ex]{0pt}{4ex}\begin{minipage}[t]{\textwidth}
			\textbf{Vorschlag: Wir schlagen die Bundessatzung vor, da alle Mitglieder der Pfarrleitung ein Stimmrecht besitzen.\\}  
		\end{minipage}              \\ \hline
		\rule[-1ex]{0pt}{4ex}\begin{minipage}[t]{\textwidth} 
			Die Mitglieder der Pfarrleitung können ihren Rücktritt nur gegenüber der Mitgliederversammlung erklären.\\
		\end{minipage}
		\\ \hline
	\end{tabular}
\end{table}

Sind alle Stellen der Pfarrleitung vakant, so dürfen deren Aufgaben von der
Diözesanleitung übernommen werden. In diesem Fall hat die Diözesanleitung die Möglichkeit eine
Stimme bei der Mitgliederversammlung wahrzunehmen.


% setze die nummerierung richtig für den Kindersenat:
\setcounter{chapter}{1}
\setcounter{section}{3}
\setcounter{subsection}{4}
\subsection{Der Kindersenat}
Der Kindersenat dient der Kindermitbestimmung in der Zeit zwischen den Mitgliederversammlungen.
In den Kindersenat können Dauermitglieder der KjG Pfarrgemeinschaft bis einschließlich 12
Jahre gewählt werden.

\begin{table}[H]
	\begin{tabular}{|l|}
		\hline
		\rowcolor[HTML]{FFCC67} 
		\rule[-1ex]{0pt}{4ex} \textbf{KjG Regensburg (11/2019)}     \hspace{0.6\textwidth} \showcounter        \\ \hline
		\rule[-1ex]{0pt}{4ex} \begin{minipage}[t]{\textwidth} 
			Die stimmberechtigten Mitglieder des Kindersenats werden auf der Mitgliederversammlung von den bis einschließlich 12 Jahre alten
                        Dauermitgliedern für die Dauer von einem Jahr gewählt.
			\rule[-1.2ex]{0pt}{0pt}
		\end{minipage}
		\\ \hline
		\rowcolor[HTML]{9AFF99} 
		\rule[-1ex]{0pt}{4ex}\begin{minipage}[t]{\textwidth}
			\textbf{Vorschlag: Wir schlagen vor "stimmberechtigt" hier zu streichen, da alle Mitglieder des Kindersenats ein Stimmrecht besitzen.\\}  
		\end{minipage}             \\ \hline
		\rule[-1ex]{0pt}{4ex}\begin{minipage}[t]{\textwidth} 
			Die Mitglieder des Kindersenats werden auf der Mitgliederversammlung von den bis einschließlich 12 Jahre alten
                        Dauermitgliedern für die Dauer von einem Jahr gewählt.
		\end{minipage}
		\\ \hline
	\end{tabular}
\end{table}


\subsubsection{Zusammensetzung und Einberufung des Kindersenats}

Der Kindersenat ist paritätisch zu besetzen, ihm gehören mindestens an:

\begin{table}[H]
	\begin{tabular}{|l|}
		\hline
		\rowcolor[HTML]{FFCC67} 
		\rule[-1ex]{0pt}{4ex} \textbf{KjG Regensburg (11/2019)}     \hspace{0.6\textwidth} \showcounter        \\ \hline
		\rule[-1ex]{0pt}{4ex} \begin{minipage}[t]{\textwidth} 
			Stimmberechtigt:
			\rule[-1.2ex]{0pt}{0pt}
		\end{minipage}
		\\ \hline
		\rowcolor[HTML]{9AFF99} 
		\rule[-1ex]{0pt}{4ex}\begin{minipage}[t]{\textwidth}
			\textbf{Vorschlag: Wir schlagen vor Stimmberechtigt" hier zu streichen, da alle Mitglieder des Kindersenats ein Stimmrecht besitzen.\\}  
		\end{minipage}             \\ \hline
		\rule[-1ex]{0pt}{4ex}\begin{minipage}[t]{\textwidth} 
			-
		\end{minipage}
		\\ \hline
	\end{tabular}
\end{table}
\begin{itemize}
  \item 2 Jungen
  \item 2 Mädchen
\end{itemize}
Die Aufgaben des Kindersenates können auch dann wahrgenommen werden, wenn nicht alle Ämter besetzt sind.
Von der Verpflichtung zur Parität sind die KjG Pfarrgemeinschaften ausgenommen,
in denen nur Mädchen oder Jungen vertreten sind.

Der Kindersenat wird regelmäßig, mindestens zweimal im Jahr, von der Pfarrleitung einberufen
und von einem Mitglied der Pfarrleitung geleitet.

\chapter*{Begründung}

Wir haben uns als Satzungsausschuss getroffen und unsere aktuelle Satzung auf Ungereimtheiten und Abweichungen zur Bundessatzung hin untersucht.
Dabei sind uns die  Punkte aufgefallen, die wir gerne in der Satzung geändert sehen, um die Satzung leichter verständlich zu machen.
Genauere Gründe könnt ihr an den jeweiligen Passagen finden.
% Ende des Dokumentes
\end{flushleft}
\end{document}
