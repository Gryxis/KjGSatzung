\documentclass[12pt]{report}
\usepackage[a4paper, left=2cm,top=2cm]{geometry}

%keine einzelnen Zeilen am Anfang/Ende:
\widowpenalty = 10000
\clubpenalty = 10000
%------------------------------------
%   Required Packages
%------------------------------------
\usepackage{xcolor}
\usepackage{graphicx}
\usepackage[pages=some]{background}
\usepackage[utf8]{inputenc}
\usepackage[english,ngerman]{babel} % deutsche Bezeichnungen und Worttrennung… 
\usepackage[T1]{fontenc} % Wird u.a. f\"ur das Trennen von W\"ortern  mit Umlauten genutzt.
\usepackage{tgbonum} %schriftarten
%------------------------------------------
%    Farben
%------------------------------------------

\definecolor{weiß}{RGB}{255,255,255}
\definecolor{kjggrau}{RGB}{119,120,123}
\definecolor{kjghellgrau}{RGB}{192,192,192}
\definecolor{kjgblau}{RGB}{54,95,145}
\definecolor{kjgtuerkis}{RGB}{0, 181,190}

%------------------------------------------
%markierung um gelöschte oder hinzugefügte sachen zu markieren
\usepackage{changes} 
%------------------------------------------
%------------------------------------------
% Start-cover als Einband 
%------------------------------------------
\backgroundsetup{
 scale=1,
 angle=0,
 color=white,
 placement=top,
 opacity = 1,
 contents= {
  \includegraphics[width=\paperwidth, height=26.80cm]{satzung_einband.png}
 }
}

%--------------------------------------
% Fußnoten mehfrach referenzieren:
% Quelle:
% https://www.macuser.de/threads/latex-fussnote-mehrfach-referenzieren.519258/
\newcommand{\footnoteremember}[2]{
  \footnote{#2}
  \newcounter{#1}
  \setcounter{#1}{\value{footnote}}
}
\newcommand{\footnoterecall}[1]{%
  \footnotemark[\value{#1}]
}
%--------------------------------------



\title{Satzung KjG Regensburg}
\author{}
\setcounter{chapter}{-1}

\begin{document}


 \setcounter{page}{1}


%alles links zentrieren und keine einrückung bei Paragraphen:
\begin{flushleft}
%Jetzt geht es mit Inhalt los:
Die Diözesankonferenz möge beschließen folgende markierte Passagen aus der Satzung zu streichen:

\subsubsection{Zusammensetzung des Diözesanauschusses}
Stimmberechtigte Mitglieder des Diözesanausschusses sind:
\begin{itemize}
  \item 4 weibliche Mitglieder der Pfarrleitungen bzw. Mitglieder einer Pfarrgemeinschaft, die von der
        Mitgliederversammlung ein Mandat erhalten haben. \deleted{Von diesen sollte mindestens eine Person
        Geistliche Leiterin sein.}
  \item 4 männliche Mitglieder der Pfarrleitungen bzw. Mitglieder einer Pfarrgemeinschaft, die von
        der Mitgliederversammlung ein Mandat erhalten haben. \deleted{Von diesen sollte mindestens eine
        Person Geistlicher Leiter sein.}
  \item Die Mitglieder der Diözesanleitung
\end{itemize}

\section*{Begründung}
Das Amt des DA-Geist wurde in den letzten Jahren nicht besetzt. Wir empfehlen eine Abschaffung des expliziten Amtes eines DA-Geist, da dies den Wahlprozess vereinfacht und eine starke spirituelle Einbringung der DA Mitglieder dadurch in keiner Weise ausgeschlossen wird
% Ende des Dokumentes
\end{flushleft}
\end{document}
