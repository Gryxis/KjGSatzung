\documentclass[12pt]{report}
\usepackage[a4paper, left=2cm,top=2cm]{geometry}

%keine einzelnen Zeilen am Anfang/Ende:
\widowpenalty = 10000
\clubpenalty = 10000
%------------------------------------
%   Required Packages
%------------------------------------
\usepackage[table,xcdraw]{xcolor} %table, xcdraw are needed for table colors
\usepackage{enumitem} %needed to style tables
\usepackage{float} %needed for positioning tables and text in correct order
\usepackage{graphicx}
\usepackage[pages=some]{background}
\usepackage[utf8]{inputenc}
\usepackage[english,ngerman]{babel} % deutsche Bezeichnungen und Worttrennung… 
\usepackage[T1]{fontenc} % Wird u.a. f\"ur das Trennen von W\"ortern  mit Umlauten genutzt.
\usepackage{tgbonum} %schriftarten
%------------------------------------------
%    Farben
%------------------------------------------

\definecolor{weiß}{RGB}{255,255,255}
\definecolor{kjggrau}{RGB}{119,120,123}
\definecolor{kjghellgrau}{RGB}{192,192,192}
\definecolor{kjgblau}{RGB}{54,95,145}
\definecolor{kjgtuerkis}{RGB}{0, 181,190}

%------------------------------------------
% Start-cover als Einband 
%------------------------------------------
\backgroundsetup{
 scale=1,
 angle=0,
 color=white,
 placement=top,
 opacity = 1,
 contents= {
  \includegraphics[width=\paperwidth, height=26.80cm]{satzung_einband.png}
 }
}

%--------------------------------------
% Fußnoten mehfrach referenzieren:
% Quelle:
% https://www.macuser.de/threads/latex-fussnote-mehrfach-referenzieren.519258/
\newcommand{\footnoteremember}[2]{
  \footnote{#2}
  \newcounter{#1}
  \setcounter{#1}{\value{footnote}}
}
\newcommand{\footnoterecall}[1]{%
  \footnotemark[\value{#1}]
}
%--------------------------------------

%--------------------------------------
% Add a counter to number the tables for the changes
% Source:
% https://tex.stackexchange.com/questions/503668/increment-a-counter
\newcounter{tablecounter}
%Define a counter to automatically increase and display the counter
\newcommand\showcounter{\addtocounter{tablecounter}{1}\thetablecounter}
%--------------------------------------


\title{Satzung KjG Regensburg}
\author{}
\setcounter{chapter}{-1}

\begin{document}

%alles links zentrieren und keine einrückung bei Paragraphen:
\begin{flushleft}
%Jetzt geht es mit Inhalt los:

Die Diözeanskonferenz möge beschließen folgende Änderungen in die Satzung zu übernehmen:

\subsubsection{Zusammensetzung der Diözesankonferenz}
Stimmberechtigte Mitglieder der Diözesankonferenz sind:
\begin{itemize}
  \item 2 Delegierte pro KjG Pfarrgemeinschaft
  \item Die Mitglieder der Diözesanleitung
\end{itemize}

Die Delegation ist folgendermaßen zu besetzen:
\begin{itemize}
  \item{ 2 Mitglieder der Pfarrleitung bzw. von Pfarrleitung oder Mitgliederversammlung
        Delegierte unterschiedlichen Geschlechts}
\end{itemize}

Von der Verpflichtung zur geschlechtergerechten Besetzung sind die KjG Pfarrgemeinschaften ausgenommen,
in denen nur Personen eines Geschlechts vertreten sind.

{\color{green}Hat eine KjG Pfarrgemeinschaft bis drei Wochen vor der Diözesankonferenz nicht die
Mitgliedsbeiträge des Vorjahres an die Diözesanstelle bezahlt, so ruht ihr Stimmrecht.

Sollte die Diözesankonferenz in der zweiten Hälfte des Jahres stattfinden,
so müssen zusätzlich zum selben Zeitpunkt mindestens 35 Prozent der Mitgliedsbeiträge des aktuellen Jahres an die Diözesanstelle
 gezahlt sein, sonst ruht ebenso ihr Stimmrecht.

Wenn das Stimmrecht einer KjG Pfarrgemeinschaft ruht, so bedeutet das, dass die von ihr Delegierten
nicht stimmberechtigt sind. Diese gelten im Sinne der Satzung als beratende Mitglieder.
\bigskip}

\chapter*{Begründung}
ToDo

% Ende des Dokumentes
\end{flushleft}
\end{document}
