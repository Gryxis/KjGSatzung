\documentclass[12pt]{report}
\usepackage[a4paper, left=2cm,top=2cm]{geometry}

%keine einzelnen Zeilen am Anfang/Ende:
\widowpenalty = 10000
\clubpenalty = 10000
%------------------------------------
%   Required Packages
%------------------------------------
\usepackage[table,xcdraw]{xcolor} %table, xcdraw are needed for table colors
\usepackage{enumitem} %needed to style tables
\usepackage{float} %needed for positioning tables and text in correct order
\usepackage{graphicx}
\usepackage[pages=some]{background}
\usepackage[utf8]{inputenc}
\usepackage[english,ngerman]{babel} % deutsche Bezeichnungen und Worttrennung… 
\usepackage[T1]{fontenc} % Wird u.a. f\"ur das Trennen von W\"ortern  mit Umlauten genutzt.
\usepackage{tgbonum} %schriftarten
%------------------------------------------
%    Farben
%------------------------------------------

\definecolor{weiß}{RGB}{255,255,255}
\definecolor{kjggrau}{RGB}{119,120,123}
\definecolor{kjghellgrau}{RGB}{192,192,192}
\definecolor{kjgblau}{RGB}{54,95,145}
\definecolor{kjgtuerkis}{RGB}{0, 181,190}

%------------------------------------------
% Start-cover als Einband 
%------------------------------------------
\backgroundsetup{
 scale=1,
 angle=0,
 color=white,
 placement=top,
 opacity = 1,
 contents= {
  \includegraphics[width=\paperwidth, height=26.80cm]{satzung_einband.png}
 }
}

%--------------------------------------
% Fußnoten mehfrach referenzieren:
% Quelle:
% https://www.macuser.de/threads/latex-fussnote-mehrfach-referenzieren.519258/
\newcommand{\footnoteremember}[2]{
  \footnote{#2}
  \newcounter{#1}
  \setcounter{#1}{\value{footnote}}
}
\newcommand{\footnoterecall}[1]{%
  \footnotemark[\value{#1}]
}
%--------------------------------------

%--------------------------------------
% Add a counter to number the tables for the changes
% Source:
% https://tex.stackexchange.com/questions/503668/increment-a-counter
\newcounter{tablecounter}
%Define a counter to automatically increase and display the counter
\newcommand\showcounter{\addtocounter{tablecounter}{1}\thetablecounter}
%--------------------------------------


\title{Satzung KjG Regensburg}
\author{}
\setcounter{chapter}{-1}

\begin{document}


%alles links zentrieren und keine einrückung bei Paragraphen:
\begin{flushleft}
%Jetzt geht es mit Inhalt los:


Die Diözesankonferenz möge beschließen die Satzung des KjG Diözesanverbandes Regensburg wie folgt anzupassen:


\subsubsection{Zusammensetzung der Pfarrleitung}

Die Pfarrleitung ist paritätisch zu besetzen, ihr gehören mindestens an:
Stimmberechtigt:
\begin{itemize}
  \item 2 Pfarrleiter
  \item 2 Pfarrleiterinnen
  \item 1 Geistlicher Leiter \footnoteremember{Berechtigung Geist}{
    Das Amt der Geistlichen Leiterin und des Geistlichen Leiters kann von Personen wahrgenommenwerden,
    die eine theologische oder religionspäd. Ausbildung abg. haben.
  }
  \item 1 Geistliche Leiterin \footnoterecall{Berechtigung Geist}
\end{itemize}
Die Aufgaben der Pfarrleitung können auch wahrgenommen werden, wenn nicht alle Stellen besetzt sind.
Von der Verpflichtung zur Parität sind die KjG Pfarrgemeinschaften ausgenommen, in
denen nur Mädchen und Frauen oder Jungen und Männer vertreten sind.

\begin{table}[H]
 \begin{tabular}{|l|}
  \hline
  \rowcolor[HTML]{FFCC67} 
  \rule[-1ex]{0pt}{4ex} \textbf{KjG Regensburg (11/2019)}     \hspace{0.6\textwidth} \showcounter        \\ \hline
  \rule[-1ex]{0pt}{4ex} \begin{minipage}[t]{\textwidth} 
   Mindestens ein Mitglied der Pfarrleitung muss voll geschäftsfähig sein.
   \rule[-1.2ex]{0pt}{0pt}
  \end{minipage}
  
  \\ \hline
  \rowcolor[HTML]{CBCEFB} 
  \rule[-1ex]{0pt}{4ex}\textbf{Bundessatzung (10/2019) - SAEA5 Beschluss 2019} \\ \hline
  \rule[-1ex]{0pt}{4ex}\begin{minipage}[t]{\textwidth} 
   Mindestens ein Mitglied der Pfarrleitung muss voll geschäftsfähig sein.\\
   Für mindestens die Hälfte der Stellen müssen beschränkt geschäftsfähige Personen (§ 106 BGB)\footnote{§106 BGB: Ein Minderjähriger, der das siebente Lebensjahr vollendet hat, ist nach Maße der §107 bis §113 in
    der Geschäftsfähigkeit beschränkt.} zur Wahl
   zugelassen werden.
   \rule[-1.2ex]{0pt}{0pt}
  \end{minipage}
  \\ \hline
  \rowcolor[HTML]{9AFF99} 
  \rule[-1ex]{0pt}{4ex}
  \begin{minipage}[t]{\textwidth}
   \textbf{Vorschlag: Wir schlagen die Bundessatzung vor, um jüngeren Interessenten aus Pfarreien den Rücken zu stärken.\\}  
    \end{minipage}           \\ \hline 
  \rule[-1ex]{0pt}{4ex}\begin{minipage}[t]{\textwidth} 
   %VERSION DER BUNDESSATZUNG
    Mindestens ein Mitglied der Pfarrleitung muss voll geschäftsfähig sein.\\
   Für mindestens die Hälfte der Stellen müssen beschränkt geschäftsfähige Personen (§ 106 BGB)\footnote{§106 BGB: Ein Minderjähriger, der das siebente Lebensjahr vollendet hat, ist nach Maße der §107 bis §113 in
    der Geschäftsfähigkeit beschränkt.} zur Wahl
   zugelassen werden.
  \end{minipage}
  \\ \hline
 \end{tabular}
\end{table}

   Die stimmberechtigten Mitglieder der Pfarrleitung werden von der Mitgliederversammlung für
zwei Jahre gewählt. Die stimmberechtigten Mitglieder der Pfarrleitung können ihren Rücktritt nur
   gegenüber der Mitgliederversammlung erklären.

Sind alle Stellen der Pfarrleitung vakant, so dürfen deren Aufgaben von der
Diözesanleitung übernommen werden. In diesem Fall hat die Diözesanleitung die Möglichkeit eine
Stimme bei der Mitgliederversammlung wahrzunehmen.

\subsubsection{Zusammensetzung des Diözesanauschusses}
Stimmberechtigte Mitglieder des Diözesanausschusses sind:
\begin{itemize}
  \item 4 weibliche Mitglieder der Pfarrleitungen bzw. Mitglieder einer Pfarrgemeinschaft, die von der
        Mitgliederversammlung ein Mandat erhalten haben. Von diesen sollte mindestens eine Person
        Geistliche Leiterin sein.
  \item 4 männliche Mitglieder der Pfarrleitungen bzw. Mitglieder einer Pfarrgemeinschaft, die von
        der Mitgliederversammlung ein Mandat erhalten haben. Von diesen sollte mindestens eine
        Person Geistlicher Leiter sein.
  \item Die Mitglieder der Diözesanleitung
\end{itemize}

Beratende Mitglieder sind:
\begin{itemize}
  \item Die Diözesanreferent*innen
\end{itemize}

Die Aufgaben des Diözesanausschusses können auch dann wahrgenommen werden, wenn nicht
alle Stellen besetzt sind.

\begin{table}[H]
 \begin{tabular}{|l|}
  \hline
  \rowcolor[HTML]{FFCC67} 
  \rule[-1ex]{0pt}{4ex} \textbf{KjG Regensburg (11/2019)}     \hspace{0.6\textwidth} \showcounter        \\ \hline
  \rule[-1ex]{0pt}{4ex} \begin{minipage}[t]{\textwidth} 
    Das Mindestalter für den Diözesanausschuss liegt bei 16 Jahren. Von den stimmberechtigten Mitgliedern
   des Diözesanausschusses, die nicht Teil der Diözesanleitung sind, muss aber mindestens
   ein Mitglied, unabhängig des Geschlechts, voll geschäftsfähig sein.
   \rule[-1.2ex]{0pt}{0pt}
  \end{minipage}
  \\ \hline
  \rowcolor[HTML]{CBCEFB} 
  \rule[-1ex]{0pt}{4ex}\textbf{Bundessatzung (10/2019)} \\ \hline
  \rule[-1ex]{0pt}{4ex}\begin{minipage}[t]{\textwidth} 
   Mitglied im Diözesanausschuss können auch Personen werden, die mindestens beschränkt geschäftsfähig (§106 BGB) sind.
   \rule[-1.2ex]{0pt}{0pt}
  \end{minipage}
  \\ \hline
  \rowcolor[HTML]{9AFF99} 
  \rule[-1ex]{0pt}{4ex}\begin{minipage}[t]{\textwidth}
   \textbf{Vorschlag: Wir wollen durch den Text bekräftigen, dass auch jüngere Personen in den DA können. Trotzdem ist uns wichtig, dass mindestens einer noch ü18 ist.\\}  
  \end{minipage}              \\ \hline
  \rule[-1ex]{0pt}{4ex}\begin{minipage}[t]{\textwidth}
    Mitglied im Diözesanausschuss können Personen werden, die mindestens beschränkt geschäftsfähig (§106 BGB)\footnote{§106 BGB: Ein Minderjähriger, der das siebente Lebensjahr vollendet hat, ist nach Maße der §107 bis §113 in
     der Geschäftsfähigkeit beschränkt.} sind. Von den stimmberechtigten Mitgliedern
    des Diözesanausschusses, die nicht Teil der Diözesanleitung sind, muss aber mindestens eine Person, unabhängig des Geschlechts, voll geschäftsfähig sein.\\
  \end{minipage}
  \\ \hline
 \end{tabular}
\end{table}

Gäste können von der Diözesanleitung oder dem Diözesanausschuss eingeladen werden.

Die Vertretungen der Pfarrgemeinschaften werden von der Diözesankonferenz für zwei Jahre
gewählt. Die Wahl ist persönlich; eine Vertretung im Diözesanausschuss ist nicht möglich. Mit
dem Wegfall der Voraussetzung für den Diözesanausschuss erlischt die Mitgliedschaft im Diözesanausschuss.

\section*{Begründung}

"Wahlrecht ohne Altersgrenzen und Partizipation von Kindern und Jugendlichen
sind seit langem Schwerpunktthemen im KjG Bundesverband. Wir sind Vorreiterin,
wenn es um Kindermitbestimmung geht und treten innerverbandlich wie politisch selbstbewusst mit unseren Forderungen auf.
Deshalb erachten wir es für konsequent, unsere Forderungen innerverbandlich stringent umzusetzen.
Dazu gehört für uns auch. alle bisher existierenden Altersgrenzen, die in der Bundessatzung vorhanden sind,
aufzuheben."

So hieß es 2019, als die Bundeskonferenz die Satzungsänderungen beschlossen hat.
Dem wollen wir folgen und stellen analog diesen Satzungsänderungsantrag.

% Ende des Dokumentes
\end{flushleft}
\end{document}
